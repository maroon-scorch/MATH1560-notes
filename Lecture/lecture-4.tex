\section{Lecture 4 (February 8)}
\subsection{Dirichlet Convolution}
\begin{definition}[Dirichlet Convolution]
Let $f, g$ are arithmetic functions. Then the Dirichlet Product/Convolution of f and g is
\begin{align*}
    (f * g)(n) &= \sum_{d_1d_2 = n} f(d_1)g(d_2)\\
    &= \sum_{d | n} f(d)g(n/d)
\end{align*}
\end{definition}

\begin{remark}
Note that $(f * g)(n)$ is also an arithmetic function, and note that it is clearly commutative.
\end{remark}

\begin{proposition}
The Dirichlet Convolution is associative.
\end{proposition}

\begin{proof}
Let $f, g, h$ be arithmetic functions
\begin{align*}
    (f * g) * h(n) &= \sum_{d_1d_2d_3 = n} f(d_1)g(d_2)h(d_3)\\
    &= f * (g * h)(n)
\end{align*}
\end{proof}

\begin{definition}[Multiplicative Identity of Dirichlet Convolution]
Let $I: \mathbb{Z}_+ \to \{0, 1\}$ be given by $I(n) = 1$ if $n = 1$, $I(n) = 0$, otherwise.\\\\
Then clearly for any arithmeitc function $I$,
\begin{align*}
    (f * I)(n) &= \sum_{d_1d_2 = n} f(d_1)I(d_2)\\
    &= f(n)I(n)\\
    &= f(n)
\end{align*}
The other way works similarly.
\end{definition}

\begin{lemma}
If $f$ is an arithmetic function with $f(1) \neq 0$, then there exists an arithmetic function $g$ such that
\[f * g = I = g * f\]
\end{lemma}

\begin{proof}
By Commutativity, we only have to show one side.\\\\
We will define $g$ recursively as
\[ g(n) = \begin{cases} 
      g(1) = 1 \\
      g(n) = \frac{-1}{f(1)} \sum_{d | n, d < n} g(d)f(n/d)
   \end{cases}
\]
Then we can see that $(f * g)(1) = f(1) \cdot g(1) = 1$.\\\\
For all $n > 1$, we have that
\begin{align*}
    \sum_{d | n} g(d)f(n/d) &= g(n)f(1) + \frac{-1}{f(1)} \sum_{d | n, d < n} g(d)f(n/d)\\
    &= \frac{-f(1)}{f(1)} \sum_{d |n , d< n} g(d) f(n/d) + \sum_{d | n, d < n} g(d)f(n/d) \tag*{Rewrite $g(n)$ with its reucrsive definition}\\
    &= 0
\end{align*}
\end{proof}

\begin{remark}
This means that the set of all arithmetic functions $f$ where $f(1) \neq 0$ form an Abelian Group with the Dirichlet Convolution as its operation.
\end{remark}

\subsection{Mobius Function}

\begin{definition}
Let $\mu: \Zbb_+ \to \{-1, 0, 1\}$ be given by
\[\mu(n) = \begin{cases}
(-1)^k, \text{if $n = p_1...p_k$, $p_i$ are distinct primes}
0, \text{otherwise}
\end{cases}\]
Note that we vacuously have that $\mu(1) = 1$.
\end{definition}

\begin{lemma}
$\mu$ is a multiplicative function.
\end{lemma}

\begin{proof}
Let $m, n  \in \mathbb{Z}_+$ with $(m, n) = 1$, then
\[m = p_1^{e_1}...p_k^{e_k}\]
\[n = q_1^{f_1}...q_l^{f_l}\]
If either $m$ or $n$ is not square-free, then clearly
\[\mu(mn) = \mu(m)\mu(n) = 0\]
Otherwise, suppose both $m$ and $n$ are square-free, then
\[m = p_1...p_k, n = q_1...q_l\]
, all distinct primes. Since $(m, n) = 1$, we also know that $(p_i, q_j) = 1$ for all choices. Then clearly
\[\mu(mn) = (-1)^{k + l} = \mu(m)\mu(n)\]
\end{proof}

\begin{lemma}
Consider the summatory function of $\mu$, then
\[f(n) = \sum_{d | n} \mu(d) = 0\]
for all $n \geq 2$
\end{lemma}

\begin{proof}
Since $\mu$ is multiplicative, then $f(n)$ is multiplicative. Now for any $n \geq 2$, so it suffices for us to check prime powers, indeed,
\[f(p^e) = \mu(1) + \mu(p) + ... + \mu(p^e) = \mu(1) + \mu(p) = 1 - 1 = 0\]
\end{proof}

\begin{definition}
Let $i: \mathbb{Z}_+ \to \{1\}$ be the constant $1$ function.
\end{definition}

\begin{lemma}
i the multiplicative inverse of $\mu$:
\[i * \mu = \mu * i = I\]
\end{lemma}

\begin{proof}
Clearly $(\mu * i)(1) = \mu(1) \cdot i(1) = 1$.\\\\
For $n > 1$, we have that
\[(i * \mu)(n) = \sum_{d | n} \mu(d) = 0\]
by the previous lemma.
\end{proof}

\begin{remark}[Summatory Functions as Convolutions]
We can turn a summatory function into a Dirichlet product in the sense that if $F(n) = \sum_{d | n} f(d)$, then $F = f * i$.
\end{remark}

\begin{remark}
If $f, g$ are multiplicative, then so is $f * g$. The proof is left as an exercise for the reader.
\end{remark}

\begin{theorem}[Mobius Inversion]
Let $F(n) = \sum_{d | n} f(d)$, then
\[f(n) = \sum_{d | n} \mu(d)F(n / d) = \mu * F\]
\end{theorem}

\begin{proof}
We take advantage of the fact that much of this is simplified by the language of abstract algebra since
\[F * \mu = (f * i) * \mu = f * (i * \mu) = f * I = f\]
\end{proof}

\begin{corollary}
If $F$ is the summatory function of $f$ and $F$ is multiplicative, then so is $f$
\end{corollary}

\begin{proof}
Since Dirichlet Convolution preserves multiplicativity and both $F$ and $\mu$ are multiplicative, $F * \mu = f$ is thus multiplicative.
\end{proof}

\begin{corollary}
This is another way to show that $\phi$ is multiplicative
\end{corollary}

\begin{proof}
Since $F(n) = \sum_{d | n} \phi(d) = n$ is multiplicative, $\phi = F * \mu$ is multiplicative.
\end{proof}

\subsection{Applications of Mobius Inversion:}

\begin{definition}
The n-th cyclotomic polynomial $\Phi_n(x)$ is the unique irreducible polynomial in $\mathbb{Z}[x]$ dividing $x^n - 1$ but does not divide $x^k - 1$ for $k < n$, then we have that
\[\Phi_n(x) = \prod_{1 \leq k \leq n, gcd(k, n) = 1} (x - e^{\frac{2\pi i k }{n}})\]
So in other words, the $n$-th cyclotomic polynomial is the product of the linear factors with root being the n-th primitive roots of unity.
\end{definition}

\begin{proposition}
\[\prod_{d | n} \Phi_d(x) = x^n - 1\]
\end{proposition}

\begin{proof}
By Mobius Inversion, if $G(n) = \prod_{d | n} g(d)$, then
\[g(n) = \prod_{d | n} G(d)^{\mu(n/ d)}\]
Thus, when $G(n) = x^n - 1$, we have that
\[g(n) = \prod_{d | n} (x^d - 1)^{\mu(n/d)}\]
, where $\mu(n / d) = 1$ only at $d = n$, so
\[g(n) = x^n - 1\]
Thus, since $G(n) = x^n - 1$, we have that
\[x^n - 1 = \prod_{d | n} g(d) = \prod_{d | n} (x^d - 1)\]
\end{proof}

{\bf Alternative Proof}
\begin{proof}
The set of roots of $x^n - 1$ form a cyclic group of order $n$, clearly each root of $\Phi_d(x)$ is also a root of $\Phi_n(x)$. Then intersection of roots of different order is empty, the union of all roots of each order dividing $n$ is the entire set of roots. Thus, the two has to equal.
\end{proof}

\begin{definition}[Dynatomic Polynomials]
Dynatomic polynomials have as roots periodic points of a polynomial.\\\\
Let K be a field, and let $f \in k[x]$ be of deg $d \geq 2$, let $f^n = f \circ ... \circ f$, f composed itself $n$ times.\\\\
We say $p \in \overline{K}$ is periodic under $f$ if
\[f^n(p) = p\]
for some $n$.
\end{definition}

\begin{example}
0 is a period point of $f(x) = x^2 - 1$, since $f(0) = -1$, $f(f(0)) = f(-1) = 0$.
\end{example}

\begin{definition}
If $n$ is the smallest integer such that $f^n(p) = p$, then $p$ has \textit{exact period} n.
\end{definition}

\begin{definition}
The $n$-th dynatomic polynomial of $f$ is
\[\Phi_{f, n}(x) = \prod_{d | n} (f^d(x) - x)^{\mu(n/d)}\]
and hence
\[\prod_{d | n} \Phi_{f, n}(x) = f^n(x) - x\]
\end{definition}

\noindent We hope that $\Phi_{f, n}(x)$ have roots that exactly the points of exact period $n$. This is false

\begin{example}
Let $f(x) = x^2 - 3/4$, then $f(x) - x = (x - 3/2)(x + 1/2)$ has roots $3/2, -1/2$.\\\\
Note that $f^2(x) - x = x^4 - 3/2x^2 - x = 3/16 = (x - 3/2)(x + 1/2)$, so we have that
\[\frac{f^2(x) - x}{f(x) - x} = (x + 1/2)^2\]
But $x = -1/2$ is a fixed point, hence a contradiction.
\end{example}

\begin{remark}
For a degree 2 polynomial, generally it has 2 distinct points of exact period 2 and 2 more points distinct from the two prior that are fixed under f.\\\\
Here, the 2 former points have collided into 1 fixed point.
\end{remark}