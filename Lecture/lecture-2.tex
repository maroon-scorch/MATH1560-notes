\section{Lecture 2 (February 1st)}

\subsection{Annoucement}
A few notes:
\begin{itemize}
    \item Homework 0 (biography) is going up tonight
    \item Lecture notes will (generally) be posted after each class
\end{itemize}

\subsection{Unique Factorization}

\begin{definition}[Common Definitions]
We will establish some elementary notations first:
\begin{itemize}
    \item We denote $a | b$ as ``a divides b" and $a \nmid b$ as ``a does not divide b".
    \item  We say that a positive integer $p \geq 2$ is \textit{prime} if its only positive divisors are 1 and p.
    \item $\mathbb{Z}_+$ denotes the set of positive integers, and let $\mathbb{N}$ be the natural numbers with 0!
    \item For a non-zero $n \in \mathbb{Z}$ and a prime p, there is a non-negative integer a such that $p^a | n$ but $p^{n+1} | n$, the number $a$ is denoted as \textit{the order of n at p} or $ord_pn$
    \item For $n = 0$, by convention we say that $ord_p 0 = \infty$
    \item Note that fairly trivially
    \[ord_p n = 0 \iff p \nmid n\]
\end{itemize}
\end{definition}

\begin{lemma}
Every non-zero integer can be written as a product of primes, except for -1. (Note that by convention the empty product is 1)
\end{lemma}

\begin{proof}
Suppose there exist some non-zero integer that can't be written as a product of primes, let N be the smallest integer greater than 2 that can't be written as a product of primes.\\\\
Clearly, N is not prime, or else $N = N$ is a valid product of primes.\\\\
Then we can write $N = a \cdot b$ where $1 < a, b < N$, but a and b can be written as product of primes, since N is the minimal number greater than 2 that can't be written as a product of primes. This is a contradiction.
\end{proof}

\begin{remark}
Hence, for all non-negative integer n, we can write
\[n = (-1)^{\epsilon} \prod_{p\ \text{pos prime}} p^{a(p)}\]
, where $\epsilon \in \{0, 1\}$ and $a(p)$ is a non-negative integer.
\end{remark}

\begin{theorem}[Unique Factorization]
For every non-zero integer n, there is a prime factorization 
\[n = (-1)^{\epsilon} \prod_{p\ \text{pos prime}} p^{a(p)}\]
where $\epsilon, a(p)$ are uniquely determined.\\\\
Moreover, $a(p) = ord_p n$
\end{theorem}

Before the proof, we recall a few things from Abstract Algebra
\begin{lemma}[The Integers are an Euclidean Domain]
If $a, b \in \mathbb{Z}$ with $b > 0$, then there exist $q, r \in \mathbb{Z}$ st.
\[a = bq + r\]
, with $0 \leq r < b$
\end{lemma}

\begin{proof}
Consider the set 
\[S = \{a - xb\ |\ x \in \mathbb{Z}\}\]
Note that S contains positive elements, (if a is positive, pick x = 0, if a is negative or zero, pick sufficiently negative value of x)\\\\
Let $r = a - qb$ be the least non-negative element of S.\\\\
Then we claim that the pair $(q, r)$ here is the pair we are looking for. To do this, we only need to show that $0 \leq r < b$, suppose for the sake of contradiction that $r \geq b$, then
\[r = a - qb \geq b\]
But this means that
\[a - (q + 1)b \geq 0\]
So r is not the smallest integer possible satisfying the constraint, contradiction.
\end{proof}

\begin{corollary}
$\mathbb{Z}$ is an Euclidean domain, with Euclidean function given by $x \mapsto |x|$.
\end{corollary}

\noindent Quick recall on what the Euclidean Domain is:
\begin{definition}[Euclidean Domain]
Let R be an integral domain, then R is said to be a Euclidean Domain if there exist a \textit{Euclidean function} $\lambda: R - \{0\} \to \mathbb{Z}_{\geq 0}$ such that
\begin{itemize}
    \item if $a, b \in R$ and $b \neq 0$, then there exists $c, d \in R$ such that
    \[a = cb + d, d = 0 \text{ or } \lambda(d) < \lambda(b)\]
\end{itemize}
\end{definition}

\begin{example}
$\mathbb{Z}$ is a Euclidean domain with $\lambda = |\cdot|$\\
$k[x]$, where $k$ is a field is an Euclidean Domain with $\lambda = deg$
\end{example}

\begin{definition}
If an ideal $I = (a)$ for some $a \in I$, then I is said to be a principal ideal.\\\\
R is said to be a principal ideal domain (PID) if every ideal in R is a principal ideal.
\end{definition}

\begin{proposition}
If R is an Euclidean Domain, then R is a PID. (ie. For any ideal $I \subset R$, there exists some $a \in R$ such that $I = aR$
\end{proposition}

\begin{proof}
If I is the zero ideal, then $I = (0)$ and is thus principal.\\\\
Now suppose I is a non-zero ideal, then there exists some non-zero element $a \in I$, we can choose a such that $\lambda(a)$ is minimal.\\\\
Now consider all $b \in I$, since R is an Euclidean Domain, we have that
\[b = qa + r\]
, where $\lambda(r) < \lambda(a)$ or $r = 0$.\\\\
If r is non-zero, this would violate the minimality of $\lambda(a)$ and be a contradiction, thus $r = 0$.\\\\
SO $b = qa$. Thus, I is a principal ideal.\\\\
Thus, R is a PID.
\end{proof}

\begin{definition}[Irreducible vs Prime]
We say $p \in R$ is \textbf{irreducible} if $a | p \implies $ a is either a unit, or an associate of p. We say $p \in R$ is \textbf{prime} if p is not a unit, p is non-zero, and $p | ab$ implies that either $p | a$ or $p | b$
\end{definition}

\begin{remark}
The zero ideal is a prime ideal. The prime elements generate prime ideals, but the zero element itself is not prime.
\end{remark}

\begin{definition}[Greatest Common Divisor]
Let R be an integral domain, then we $d \in R$ is said to be the gcd of $a, b \in R$ if
\begin{itemize}
    \item i) d divides a and d divides b
    \item ii) If $d^\prime | a$ and $d^\prime | b$, then $d^\prime | d$
\end{itemize}
We denote $(a, b)$ as the gcd of a and b, and in fact in the integers,
\[dR = aR + bR\]
gcd's are also only unique up to units. For the integers, we will refer the gcd to only the positive gcd.\\\\
Aside for Ring Theory enthusiasts, GCD Domains are a class of rings more general than PIDs and UFDs.
\end{definition}

\noindent There are two important properties of PIDs:
\begin{itemize}
    \item 1) Nonunit irreducible elements are exactly the prime elements
    \item 2) GCDs always exist in PIDs
\end{itemize}

\noindent We are now ready to prove unique factorization in $\mathbb{Z}$, after the lemma.

\begin{lemma}
Suppose p is a prime, and $a, b \in \mathbb{Z}$, then $ord_p(ab) = ord_p(a) + ord_p(b)$
\end{lemma}

\begin{proof}
If one of a and b is 0, that case is trivial.\\\\
So WLOG, we can assume that $a, b \neq 0$. Let $\alpha = ord_p(a)$ and $\beta = ord_p(b)$, then
\[a = p^{\alpha}*c,\ p \nmid c\]
\[b = p^\beta*d,\ p \nmid d\]
So we have that
\[ab = p^{\alpha + \beta}(dc)\]
Note that $p \nmid dc$ because $p \nmid d$ and $p \nmid c$ (contrapositive of definition of prime), so $ord_p(ab) = \alpha + \beta$.\\\\
Note that we used the fact that the non-unit irreducible element is prime in a PID (so the definition of prime in $\mathbb{Z}$ aligns with that of prime in PID).
\end{proof}

\begin{theorem}[The Fundamental Theorem of Arithmetic]
$\mathbb{Z}$ is a UFD.
\end{theorem}

\begin{proof}
Recall that for a non-zero $n \in \mathbb{Z}$, we can write
\[n = (-1)^{\epsilon} \prod_{p\ \text{pos prime}} p^{a(p)}\]
, where $\epsilon \in \{0, 1\}$ and $a(p) \geq 0$.\\\\
This is the existence part, now we will prove the uniqueness part.\\\\
Given a positive prime q, we can take $ord_q$ of both sides of the expressions:
\begin{align*}
    ord_q(n) &= \epsilon \cdot ord_q(-1) + \sum_p a(p) ord_q(p) \\
    &= 0 + \sum_p a(p) ord_q(p)\\
    &= a(q)ord_q(q) \tag*{$ord_q(p) = 0$ for all $p \neq q$}\\
    &= a(q) \cdot 1\\
    &= a(q)
\end{align*}
Since the positive prime q is arbitrarily chosen, the factorization has to be unique.
\end{proof}
