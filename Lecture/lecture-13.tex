\section{Lecture March 22nd}

\subsection{Discriminants of bases, Vandermonde determinants}

Let $K = \Qbb(\theta)$ be a number field of degree $n$, let $\{\alpha_1, ..., \alpha_n\}$ be a $\Qbb$-basis of $K$, and let $\sigma_i: K \to \Cbb$, $1 \leq i \leq n$ be the embeddings of $K$ into $\Cbb$.\\

\noindent The discriminant of $\{\alpha_1, ..., \alpha_n\}$ is 
\[\Delta[\alpha_1, ..., \alpha_n] = det(\begin{bmatrix} \sigma_1(\alpha_1) & ... & \sigma_1(\alpha_n)\\
\vdots & \ddots & \vdots \\
\sigma_n(\alpha_1) & ... & \sigma_n(\alpha_n)
\end{bmatrix})^2\]

\noindent If $\{\beta_1, ..., \beta_n\}$ is another basis, then for all $1 \leq k \leq n$,
\[\beta_k = \sum_{i = 1}^n c_{ik} \alpha_i, c_{ik} \in \Qbb\]
, where $det(C_{ik}) \neq 0$.\\

\noindent From the homework, we will prove that
\[\Delta[\beta_1, ..., \beta_n] = det(C_{ik})^2 \cdot \Delta[\alpha_1, ..., \alpha_n]\]

\begin{definition}
A square Vandermonde matrix is a matrix of the form
\[V = \begin{bmatrix}
1 & t_1 & ... & t_1^{n-1}\\
1 & t_2 & ... & t_2^{n-1}\\
\vdots & \vdots & \ddots & \vdots\\
1 & t_n & ... & t_n^{n-1}\\
\end{bmatrix}\]
\end{definition}

\begin{proposition}
The determinant of V is
\[det(V) = \prod_{1 \leq i < j \leq n} (t_i - t_j)\]
\end{proposition}

\begin{proof}
Let $D = \prod_{1 \leq i < j \leq n} (t_j - t_i)$.\\\\
On one hand we know that $det(V) = 0$ when $t_i = t_j$ for some $i \neq j$ since the rows won't be linearly independent, $det(V)$ (as a polynomial in $t_1, ..., t_n$) is disivible by $(t_j - t_i)$, $i < j$ since they are roots of the polynomial.\\\\
The above argument works if we use the Factor Theorem by treating this as a polynomial in terms of $t_i$ and treating everything else as constants.\\\\
What is the degree of this polynomial??\\\\
Well, inductively we can show that the degree of this polynomial is the triangle numbers, so for $n \times n$ matrix $V$, it is the $n-1$-th triangle number
\[deg(det(V)) = \sum_{i = 1}^{n-1} i = \frac{(n-1)n}{2}\]
The degree of $D$ is also $\frac{(n-1)n}{2}$, since by combinatorics we have ${n \choose 2} = \frac{(n-1)n}{2}$ choices for $i$ and $j$.\\\\
Hence, $det(V)$ is a scalar multiple of $D$. Now both terms are monomial terms with either 1 or -1 as coefficients, so just pick one term in both and see if they have the same sign, which they do, so we conclude that they are the same polynomial.
\end{proof}

\begin{theorem}[pg. 42 of Stewart and Tall]
The discriminant of any $\Qbb$-basis for $K$ is rational and non-zero.
\end{theorem}

\begin{proof}
By the fact proven from the homework, it suffices to proce this for the basis $\{1, \theta, .., \theta^{n-1}\}$.\\\\
Now we take $t_i = \theta_i = \sigma_i(\theta)$, to get that
\[\Delta[1, \theta, ..., \theta^{n-1}] =\prod_{i < j} (\theta_i - \theta_j)^2\]
\[ = disc(minpoly_\Qbb(\theta)) \in \Qbb^*\]
, this product is fixed by all $\sigma_i$, so it's in the fixed field of all of them, which is $\Qbb$.\\\\
It is non-zero because $\theta_i \neq \theta_j \iff i \neq j$.
\end{proof}

\begin{example}
Suppose $K = \Qbb(\sqrt{5})$, pic basis $\{1, \sqrt{5}\}$, therefore we have that
\[\Delta[1, \sqrt{5}] = det(\begin{bmatrix} 1 & \sqrt{5} \\ 1 & -\sqrt{5}\end{bmatrix})^2 = (-2\sqrt{5})^2 = 20\]
Another basis is $\{1, \frac{1 + \sqrt{5}}{2}\}$, then
\[\Delta[1, \frac{1 + \sqrt{5}}{2}] = det(\begin{bmatrix} 1 & (1 + \sqrt{5})/2 \\ 1 & (1-\sqrt{5})/2 \end{bmatrix})^2 = 5\]
And we see that they differ by a square.\\\\
Suppose $K = \Qbb(2^{1/3})$, a basis is $B = \{1, 2^{1/3}, 2^{2/3}\}$, then
\[\Delta[B] = det(\begin{bmatrix} 
1 & 2^{1/3} & 2^{2/3}\\
1 & \omega 2^{1/3} & \omega^2 2^{2/3}\\
1 & \omega^2 2^{1/3} & \omega^4 2^{2/3}
\end{bmatrix})^2\]
\end{example}

\subsection{Algebraic Integers}

\begin{definition}
A complex number is an algebraic integer if it is a root of a \textbf{monic polynomial with integer coefficients}.\\\\
We denote the set of algebraic integers by $\overline{\Zbb}$, and clearly
\[\overline{\Zbb} \subset \overline{\Qbb}\]
\end{definition}

\begin{example}
{\bf Examples: }\\\\
$\sqrt{2}$ is an algebraic integer, just take $x^2 - 2$.\\\\
$\frac{1 + \sqrt{5}}{2}$ is an algebraic integer, take $x^2 - x - 1 = 0$.\\\\
{\bf Non-Examples: }\\\\
$\frac{22}{7}$ is not an algebraic integer, mostly because the irreducible polynomial $7x - 22$ has to divide any polynomial whose root is $\frac{22}{7}$
\end{example}

\noindent {\bf Key Algebra Fact:}

\begin{proposition}[Gauss's Lemma]
If $f(x) \in \Zbb[x]$ is a monic polynomial with $f(x) = g(x)h(x)$, where $g(x), h(x)$ are monic polynomials in $\Qbb[x]$, then $g(x), h(x)$ are in fact in $\Zbb[x]$. (Gauss's Lemma)
\end{proposition}

\begin{proposition}[Equivalent Definition of Algebraic Integers]
An algebraic number $\theta$ is an algebraic integer if and only if $minpoly_\Qbb(\theta)$ has integer coefficients!
\end{proposition}

% 44 - 45 of Stewart and Tall 