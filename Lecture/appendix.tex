\section{Appendix}
This section contains some important theorems that were covered in Problem Sets of the course but were not covered during lecture.

\subsection{Dedekind Domains are 2-generated Ideal Domains}

In this section, we will show that every integral ideal of Dedekind Domain $\mathcal{O}_K$ can be generated by 2 elements it contains.

\begin{lemma}
Suppose $\afrak, \bfrak$ are comaximal, then $\afrak^n$ and $\bfrak$ are also comaximal in $\mathcal{O}_K$ for all integers $n \geq 1$
\end{lemma}

\begin{proof}
Let's induct on $n$, clearly when $n = 1$, this is trivial.\\\\
Now suppose our inductive hypothesis is true until $n = k$, meaning that $\afrak^k + \bfrak = (1)$. Then we wish to show that $\afrak^{k+1} + \bfrak = (1)$.\\\\
Indeed, it suffices for us to show that $1 \in \afrak^{k+1} + \bfrak$, since the other direction is trivial. Since $\afrak + \bfrak = 1$, there exist some element $a \in \afrak, b \in \bfrak$ such that $a + b = 1$.\\\\
Since $\afrak^{k} + \bfrak = (1)$, this means that there exist some element $x \in \afrak^k, y \in \bfrak$ such that $x + y = 1$, now
\[1 = (a + b)(x + y) = ax + bx + by + ay = ax + (bx + by + ay)\]
Since $a \in \afrak, x \in \afrak^{k}$, clearly $ax \in \afrak^{k+1}$. Since $b \in \bfrak, y \in \bfrak$, we have that $bx + by + ay \in \bfrak$, thus $1$ is the sum of an element from $\afrak^{k+1}$ and $\bfrak$, so $1 \in \afrak^{k+1} + \bfrak$.
\end{proof}

\begin{corollary}
Suppose $\afrak, \bfrak$ are comaximal, then $\afrak^m$ and $\bfrak^n$ are also comaximal for any integers $n, m \geq 1$
\end{corollary}

\begin{proof}
Apply Lemma 23.1 for the first time shows us that $\afrak^m$ and $\bfrak$ are comaximal. Applying Lemma 23.1 the second time shows us that $\afrak^m$ and $\bfrak^n$ are comaximal.
\end{proof}

\begin{corollary}
Suppose $\pfrak_1, ..., \pfrak_n$ are pairwise comaximal, then $\pfrak_1^{e_1}, ..., \pfrak_n^{e_n}$ are pairwise comaximal for any integer $e_1, ..., e_n \geq 1$
\end{corollary}

\begin{proof}
Apply Corollary 23.2 on each pair.
\end{proof}

\begin{lemma}
Let $\afrak, \bfrak, \efrak$ be pairwise comaximal ideals, then $\afrak \bfrak$ and $\efrak$ are comaximal.
\end{lemma}

\begin{proof}
Again, it suffcies for us to show that $1 \in \afrak \bfrak + \efrak$. Indeed, since $1 \in \afrak + \efrak$, $1 \in \bfrak + \efrak$, we have that there exist $a \in \afrak, b \in \bfrak, e_1, e_2 \in \efrak$ such that
\[a + e_1 = 1, b + e_2 = 1\]
Now we have that
\[1 = (a + e_1)(b + e_2) = ab + (ae_2 + be_1 + e_1e_2)\]
Then clearly $ab \in \afrak \bfrak$ and $ae_2 + be_1 + e_1e_2 \in \efrak$, so $1 \in \afrak \bfrak + \efrak$.
\end{proof}

\begin{theorem}
Let $I$ be an integral ideal of $\mathcal{O}_K$, then $I$ is either principal or generated by $2$ elements.
\end{theorem}

\begin{proof}
If $I$ is the zero ideal, then $I = (0)$ is already principal.\\\\
If $I = \mathcal{O}_K$ the entire ring, then $I = (1)$ is also principal.\\\\ 
Now if $I$ is a non-zero non-unit ideal, then we can choose some $a \in I$ where $a \neq 0$.\\\\
Since we proved that unique factorization of ideals exist in $\mathcal{O}_K$, we can write
\[I = \pfrak_1^{e_1} ... \pfrak_r^{e_r}\]
as $I$'s unique factorization of prime ideals $\pfrak_1, ..., \pfrak_r$.\\\\
Now since $a \in I$, we have that $(a)$, the principal ideal generated by $a$, is also a subset of $I$. So $(a) \subset I$. Moreover, we proved in lecture that $(\afrak) \subset I$ if and only if $I | (a)$, so in other words, the unique prime factorization of $(a)$ would be
\[(a) = \pfrak_1^{f_1} ... \pfrak_r^{f_r} \qfrak_1^{g_1} ... \qfrak_s^{g_s}, f_1 \geq e_1, ..., f_r \geq e_r\]
Now, we claim that there exist some element $b \in I$ such that its unique prime factorization is
\[(b) = \pfrak_1^{e_1} ... \pfrak_r^{e_r} \afrak_1^{h_1} ... \afrak_t^{h_t}\]
, where each $\afrak_i$ is distinct from any of $\qfrak_1, ..., \qfrak_s$.\\\\
Indeed, we claim it suffices for us to find an element $b \in I$ such that $b \notin \pfrak_1^{e_1 + 1}$, ... $b \notin \pfrak_r^{e_r + 1}$ and $b \notin \qfrak_1$, ..., $b \notin \qfrak_s$.\\\\
This is because since $b \in I$, we have that $I | (b)$, so $b$ has prime factors $\pfrak_1^{e_1} ... \pfrak_r^{e_r}$, but this any of those $\pfrak_i$ has more than $e_i$ in their exponent, this should imply that $b \in \pfrak_i^{e_i + 1}$ and hence a contradiction. Thus, the order of each $\pfrak_i$ for $(b)$ is $e_i$.\\\\
Moreover, if $(b)$'s unique factorization shares any common factor with at least one of $\qfrak_1, ..., \qfrak_s$, say $\qfrak_j$, then it automatically means that $b \in \qfrak_j$, so we again have a contradiction. Thus, the order of each $\qfrak_j$ in $(b)$ is $0$.\\\\
Now we want to prove we can find this said $b$. Indeed, clearly for each $\pfrak_i$, $\pfrak_i^{e_i + 1} \neq \pfrak_i^{e_i}$ (or else factorization won't be unique), so we pick some element $b_i \in \pfrak_i^{e_i} - \pfrak_i^{e_i + 1}$. Now let $b_{r + 1}, ..., b_{r + s}$ all be $1$ respectively
Let $\bfrak_1 = \pfrak_1^{e_1 + 1}, ..., \bfrak_r = \pfrak_r^{e_r + 1}, \bfrak_{r+1} = \qfrak_1, ..., \bfrak_{r+s} = \qfrak_s$, then for each $\bfrak_i$, Lemma 23.4 tells us that $\bfrak_i$ and $\prod_{j \neq i} \bfrak_j$ are comaximal, so there exist some $x_i \in \prod_{j \neq i} \bfrak_j$ and $y_i \in \bfrak_i$ such that $x_i + y_i = 1$.\\\\
Now let
\[b = \sum_{i = 1}^{r+s} b_i x_i \]
We claim that this is the $b$ we are looking for. Indeed, clearly each $b_i x_i \in \prod_{i = 1}^{r + s} \bfrak_i$, and since $I$ divides $\prod_{i = 1}^{r + s} \bfrak_i$, we have that it is a subset of $I$, so $b \in I$.\\\\
Now, we wish to show that $b$ is not in any of the $\bfrak_i$. Indeed, it suffices for us to show that $b$ does not vanish when quotienting it under $\bfrak_i$, indeed, for any arbitrary $\bfrak_k$
\begin{align*}
    b &\equiv \sum_{i = 1}^{r+s} b_i x_i \mod \bfrak_k \\
    &\equiv b_k x_k \mod \bfrak_k \tag*{Every other $x_i$ is contained in $\bfrak_k$}\\
    &\equiv b_k(1 - y_k) \mod \bfrak_k \tag*{Recall $x_k + y_k = 1$, $y_k \in \bfrak_k$}\\
    &\equiv b_k \mod \bfrak_k
\end{align*}
Now if $k \in [1, r]$, then we established earlier that we picked $b_k$ so that $b_k \in \pfrak_k^{e_k} - \pfrak_k^{e_k + 1}$, so $b_k \notin \bfrak_k$. If $k \in [r + 1, r + s]$, then $b_k = 1$ is clearly not in $\bfrak_k$ since it's not the unit ideal.\\\\
Thus, we have found our desired $b$ (Note we basically just gave a constructive use of the CRT)\\\\
Now we claim that $I = (a, b)$. Indeed, we first note that the following two product of prime ideals are comaximal
\[(i) \pfrak_1^{f_1 - e_1} ... \pfrak_r^{f_r - e_r} \qfrak_1^{g_1} ... \qfrak_s^{g_s}\]
\[(ii) \afrak_1^{h_1} ... \afrak_t^{h_t}\]
Indeed, from Corollary 24.3 and the fact that $\pfrak_1, ... \pfrak_r, \qfrak_1, ..., \qfrak_s, \afrak_1, ..., \afrak_t$ are pairwise distinct prime ideals and thus pairwise comaximal, so we have that the list
\[\pfrak_1^{f_1 - e_1}, ..., \pfrak_r^{f_r - e_r}, \qfrak_1^{g_1}, ..., \qfrak_s^{g_s}, \afrak_1^{h_1}, ..., \afrak_t^{h_t}\]
is pairwise comaximal. Then repeated applications of Lemma 23.4 shows us that $(i)$ and $(ii)$ are comaximal ideals.\\\\
Thus, we have that
\[\pfrak_1^{f_1 - e_1} ... \pfrak_r^{f_r - e_r} \qfrak_1^{g_1} ... \qfrak_s^{g_s} + \afrak_1^{h_1} ... \afrak_t^{h_t} = (1)\]
Now multiplying both sides by $I = \pfrak_1^{e_1} ... \pfrak_r^{e_r}$ and recall that we showed that ideal products are distributive over addition, we have that
\[\pfrak_1^{e_1} ... \pfrak_r^{e_r}(\pfrak_1^{f_1 - e_1} ... \pfrak_r^{f_r - e_r} \qfrak_1^{g_1} ... \qfrak_s^{g_s} + \afrak_1^{h_1} ... \afrak_t^{h_t}) =  \pfrak_1^{f_1} ... \pfrak_r^{f_r} \qfrak_1^{g_1} ... \qfrak_s^{g_s} + \pfrak_1^{e_1} ... \pfrak_r^{e_r} \afrak_1^{h_1} ... \afrak_t^{h_t}\]
\[= (a) + (b)\]
\[\pfrak_1^{e_1} ... \pfrak_r^{e_r}(\pfrak_1^{f_1 - e_1} ... \pfrak_r^{f_r - e_r} \qfrak_1^{g_1} ... \qfrak_s^{g_s} + \afrak_1^{h_1} ... \afrak_t^{h_t}) = \pfrak_1^{e_1} ... \pfrak_r^{e_r}(1) = \pfrak_1^{e_1} ... \pfrak_r^{e_r} = I\]
Thus, we have that
\[I = (a) + (b) = (a, b)\]
Thus, $I$ is a $2$-generated ideal.
\end{proof}

\subsection{Infinitude of Primes in $\mathcal{O}_K$}

In this section, we will show that $\mathcal{O}_K$ has infinity many prime ideals. We accomplish this by mimicking Euclid's Infinitude of Prime argument onto $\mathcal{O}_K$.

\begin{theorem}
    $\mathcal{O}_K$ has infinitely many prime ideals.
\end{theorem}

\begin{proof}
First we will show that $\mathcal{O}_K$ has at least 1 non-zero prime ideal. Indeed, this just comes from the fact that the maximal ideal, which is prime, of $\mathcal{O}_K$ is not the zero ideal. This is because $\mathcal{O}_K$ contains some non-zero non-unit elements (for example $2$ is not a unit in $\mathcal{O}_K$ since its inverse $1/2$ is not an algebraic integer), and that from Zorn's Lemma, every non-unit element in a ring is contained in some maximal ideal.\\\\
It now suffices for us to show that $\mathcal{O}_K$ has infinitely many non-zero prime ideals. Indeed, suppose for the sake of contradiction that $\mathcal{O}_K$ only has finitely many non-zero prime ideals, say $\pfrak_1, ..., \pfrak_n$.\\\\
Now we claim that for each of these $\pfrak_i$, we can find some non-zero rational integer $a_i \in \pfrak_i$. Indeed, take any non-zero algebraic integer $r_i \in \pfrak_i$, then $r_i$ is the root of a minimal integer polynomial $f(x) \in \Zbb[x]$, we can split $f(x)$ into $f(x) = g(x) + c$ where $g(0) = 0$ and $c$ is the constant term of $f(x)$. $c$ has to be non-zero or else $f(x)$ wouldn't be minimal.\\\\
Then we note that
\[g(r_i) = -c\]
But since $g(x)$ is an integer polynomial, $g(r_i)$ is just the integral sum of powers of $r_i$, which is closed in the ideal $\pfrak_i$, so $-c \in \pfrak_i$ is a non-zero integer.\\\\
Moreover, since additive inverse is closed in an ideal, we can take $a_i \in \pfrak_i$ to be positive.\\\\
Now, consider the term $[\prod_{j = 1}^n a_j] + 1 \in \mathcal{O}_K$. It is certainly not contained in any of the $\pfrak_i$ because $a_i \in \pfrak_i$, so in particular it is not projected to zero when modding $\mathcal{O}_K$ by $\pfrak_i$:
\[[\prod_{j = 1}^n a_j] + 1 \equiv 1 \mod \pfrak_i\]
However, we recall that we have proved in a previous homework that $[\prod_{j = 1}^n a_j] + 1$ is a unit in $\mathcal{O}_K$ if and only if $N_{K/\Qbb}([\prod_{j = 1}^n a_j] + 1) = \pm 1$, but $[\prod_{j = 1}^n a_j] + 1$ is an integer and is greater than $1$ since we chose $a_i$'s to all be positive, so
\[N_{K/\Qbb}([\prod_{j = 1}^n a_j] + 1) =  ([\prod_{j = 1}^n a_j] + 1)^{[K:\Qbb]} > 1\]
Thus,$[\prod_{j = 1}^n a_j] + 1$ is not a unit in $\mathcal{O}_K$. But then Zorn's Lemma tells us that every non-unit element is contained in some maximal ideal, which is prime. But this means that $[\prod_{j = 1}^n a_j] + 1$ is contained in one of the $\pfrak_1, ..., \pfrak_n$, so we have a contradiction.\\\\
Thus, we conclude that $\mathcal{O}_K$ has infinitely many prime ideals.
\end{proof}
