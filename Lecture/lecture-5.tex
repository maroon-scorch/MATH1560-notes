\section{Lecture 5 (February 10)}

\subsection{Linear Congruence:}
Recall that for $m \in \Zbb_+$, $a, b \in \Zbb$, the linear congruence
\[ax \equiv b\ (mod\ m)\]
has a solution if and only if $(a, m) | b$.\\\\
The question is how does one find a solution?

\begin{proposition}
The following algorithm would work, given $ax \equiv b\ (mod\ m)$
\begin{itemize}
    \item 1) Divide all terms in the congruence by $d = (a, m)$
    \item 2) If Step 1 yields
    \[a^\prime x \equiv b^\prime\ mod(m^\prime)\]
    with $(a^\prime, m^\prime) = 1$, then $d^\prime = (a^\prime, b^\prime)$ is a unit mod $m^\prime$, so then we have that
    \[\frac{a^\prime}{d^\prime} x \equiv \frac{b^\prime}{d^\prime} mod(m^\prime)\]
    \item 3) Let $a^{\prime \prime} x \equiv b^{\prime \prime} mod(m^\prime)$ be the result so far, we can replace $b^{\prime \prime}$ by $b^{\prime \prime} + km^{\prime}$ such that $(a^{\prime \prime}, b^{\prime \prime} + k m^{\prime}) > 1$ allows Step 2 to be repeated so that we can find
    \[|a^{\prime \prime \prime}| < |a^{\prime \prime}|\]
    This process has to terminate since the absolute value of $a$ is decreasing in each step.
\end{itemize}
\end{proposition}

\begin{example}
Consider $10x \equiv 6\ (mod\ 14)$.\\\\
Note that $(10, 14) = 2$, so Step 1 gives
\[5x \equiv 3\ mod\ 7\]
Step 2 is redundant.\\\\
Then, consider the integer $3 + 7k$, and we want to fjnd one divisible by $5$, so clearly $3 + 7 \cdot 1 = 10$ works, so
\[5x \equiv 10\ mod\ 7\]
So we have that
\[x \equiv 2\ mod\ 7\]
\end{example}

\subsection{Simultaneous Congruence}

\begin{theorem}[Chinese Remainder Theorem - Classical Version]
Suppose that $m = m_1m_2...m_t$ and that $(m_i, m_j) = 1$ for all $i \neq j$, let $b_1, ..., b_t$ be integers and consider the system of congruences
\[x \equiv b_1 mod(m_1)\]
\[\vdots\]
\[x \equiv b_t mod(m_t)\]
Then this system always has solutions, and any two solutions differ by a multiple of $m$.
\end{theorem}

\begin{proof}

\end{proof}

\begin{remark}[Chinese Remainder Theorem - Modern Version]
Let $\psi: \Zbb \to \Zbb/m_1\Zbb \times ... \times \Zbb/m_k \Zbb$
be given by
\[\psi(n) = (n\ mod(m_1), ..., n\ mod(m_t)\]
Then the CRT tells us that $\psi$ is surjective. Moreover, the kernel of $\psi$ is clearly $m\Zbb$. Then by the first isomorphism theorem, we have that
\[\Zbb/m\Zbb \cong \Zbb/m_1\Zbb \times ... \times \Zbb/m_k \Zbb \]
It follows that if $U(m)$ is the unit group of $\Zbb/m\Zbb$, then
\[U(m) \cong U(m_1) \times ... \times U(m_t)\]
\end{remark}

\subsection{Structure of Unit Groups}

\begin{theorem}[Lagrange's Theorem]
If G is a finite group, then for every subgroup H of G, $|H|$ divides $|G|$.
\end{theorem}

\begin{corollary}
If G is a finite group of order $n$ and $a \in G$, then $a^n = e$, where e is the identity element.
\end{corollary}

\begin{theorem}[Euler's Theorem]
For any $a \in \Zbb$ with $(a, m) = 1$, we have
\[a^{\phi(m)} = 1\ mod(m)\]
\end{theorem}

\begin{proof}
Clearly in $\Zbb/m\Zbb$, for any $a \in \mathbb{Z}$ coprime to $m$, the order of $a$ is $\phi(m)$, then by Lagrange's Theorem we have that
\[a^{\phi(m)} = 1\ mod(m)\]
\end{proof}

We present an alternative proof of Euler's Theorem:

(Skip for now)

\subsection{Polynomial Training}
We will be studying roots of polynomials over $\Zbb/m\Zbb$ for various $m$, especially polynomials of the form $x^d - a$ and the case where $m = p$ is prime. We may also touch on roots in $\overline{\mathbb{F}_p}$

\begin{proposition}
If $p \nmid d$, then the polynomial $x^d - a \in (\Zbb/p\Zbb)$ $a \neq 0$ has exactly $d$ roots in some extension of $\Zbb/p\Zbb$.\\\\
Conversely, if $p | d$, then there are fewer than $d$ roots in any extension of $\mathbb{F}_p = \Zbb/m\Zbb$.
\end{proposition}

We will prove the proposition above by consider the following:

\begin{proposition}
A non-zero polynomial $f \in K[x]$ is separable if and only if $gcd(f, f^\prime) = 1$
\end{proposition}

\begin{proof}
Suppose that a polynomial $f \in K[x]$ is separable, now suppose they are not coprime, then there exists some gcd $g(x)$ that divides both. Moreover, let $r$ be the root of $g(x)$, then
then clearly
\[f(x) = (x - r)h(x)\]
, and $h(x)$ does not have root $r$ since $f$ is separable, so moreover,
\[f^{\prime}(x) = (x - r)h^{\prime}(x) + h(x)\]
Since $g(r) = 0$, $f^{\prime}(x) = h(r) = 0$, but $h(x)$ does not have root $r$. Hence a contradiction.\\\\
Conversely suppose that $gcd(f, f^\prime) = 1$, then $f$ being not separable means that it has a root $r$ of multiplicity at least 2, then
\[f(x) = (x - r)^2h(x)\]
and
\[f^\prime(x) = 2(x-r)h(x) + (x-r)^2h^\prime(x)\]
clearly also has root r, so their gcd is not 1. Hence a contradiction.
\end{proof}

{\bf Proof of Proposition 1}
\begin{proof}
Let $K = \Zbb/m\Zbb$, then clearly when $p \nmid d$, we have that
\[f(x) = x^d - a, f^\prime(x) = dx^{d-1} \neq 0\]
They don't share any common roots since $a \neq 0$, so they must be separable as they have gcd 1.\\\\
If $p | d$, then 
\[f^{\prime}(x) = dx^{d-1}  = 0\]
So $gcd(f(x), f^\prime(x)) = gcd(f(x), 0) = f(x)$, so it's not separable.
\end{proof}

Now the second proposition we will use is (Prop. 4.1.2 from textbook):

\begin{proposition}
If $p$ is a prime, and if $d | p - 1$, then the polynomial $x^d - 1 \in (\Zbb/p\Zbb)[x]$ has exactly d roots in $\Zbb/p\Zbb$.
\end{proposition}

\begin{proof}
We will use Fermat's Little Theorem which is Euler's Theorem in the case $m = p$ is a prime.\\\\
Indeed, using Fermat's little theorem, every non-zero element if $\Zbb/p\Zbb$ is a root to
\[x^{p - 1} - 1 = 0\]
So $x^{p-1} - 1$ is separable. Since $d | p - 1$, we have that $(x^d - 1) | (x^{p-1} - 1)$, so all roots of $x^d - 1$ are separable and there's $d$ of them in $\Zbb/p\Zbb$.
\end{proof}

\begin{corollary}
$G = (\Zbb/p\Zbb)^{\times}$ is cyclic
\end{corollary}

\begin{proof}
For $d | p - 1$, consider the number of elements of $G$ of order $d$, which we will denote as $\psi(d)$, then Proposition 2 tells us that
\[\sum_{c | d} \psi(d) = d\]
Since the roots of $x^d - 1$ for all $d | n$ partitions the roots of $\Zbb/p\Zbb$.\\\\
We also note that this is the identity function.\\\\
Using the Mobius Inversion, we have that since $\psi * i = id$, $\psi = id * \mu$
\[\psi(d) = \sum_{c | d} \mu(c) \cdot {d/c}\]
We note that the right hand side is just $id * \mu$, and this is just $\phi$ since $id = \phi * i$ and $\mu$ and $i$ are inverses, so we have that
\[\psi(d) = \phi(d)\]
, so we have that $\psi(d) = \phi(d)$ for all $d | p - 1$.\\\\
In particular, if $p > 2$, then $\psi(p - 1) = \phi(p - 1) > 1$. So there exists an element of order $p - 1$ in $G$.
\end{proof}