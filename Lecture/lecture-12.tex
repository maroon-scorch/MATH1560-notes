\section{Lecture March 15th}

\subsection{Recall: Number Fields}

\begin{definition}
A complex number $\alpha$ is called algebraic if it's algebraic over $\Qbb$.
\end{definition}

\begin{proposition}
The set $\overline{\Qbb}$ form a subfield of the complex numbers.
\end{proposition}

\begin{definition}
A number field is a subfield $k$ of $\Cbb$ whose degree over $\Qbb$ is finite. Using the Primitive Element Theorem, $k$ being a number field implies $k = \Qbb(\alpha)$ for some $\alpha \in \overline{\Qbb}$, they are actually equivalent definitions.
\end{definition}

\begin{remark}
Crux of the proof of Primitive Element Theorem: Suppose $K = K_1(\alpha, \beta)$, then $K = K_1(\theta)$, and we can find $\theta$ as a function of $\alpha$ and $\beta$.\\\\
Write the minimal polynomial of $\alpha, \beta$ over $k_1$:
\[(t - a_1)...(t- a_n), a_i \in \overline{\Qbb}, a_1 = \alpha\]
\[(t - b_1)...(t- b_m), b_i \in \overline{\Qbb}, b_1 = \beta\]
Irreducible polynomials in characteristic zero are separable, so we have that $a_1, ..., a_n$ are distinct, and $b_1, ..., b_m$ are distinct.\\\\
Hence for each $i$ and each $k \neq 1$, there exist at most one $x \in K$ such that
\[a_i + xb_k = a_1 + xb_1, ie. x = (a_i - \alpha)(b_1 - b_k)^{-1}\]
Since there are only finitely many of these equations, we can choose some non-zero $c \in k_1$ such that
\[a_i + cb_k \neq a_1 + cb_1, \forall 1 \leq i \leq n, 2 \leq k \leq m\]
Define $\theta = \alpha + c \beta$, then we claim that
\[K_1(\alpha, \beta) = K_1(\theta)\]
This part is done in Ireland and Stewart.
\end{remark}

\begin{example}[Page 35]
The Primitive Element Theorem actually gives us a way to find the primitive element. For example, take $K = \Qbb(\sqrt{2}, 5^{1/3})$, and let $\alpha_1 = \sqrt{2}, \beta_1 = 5^{1/3}$.\\\\
We have that $\alpha_2 = -\sqrt{2}$, $\beta_2 = \zeta_3 5^{1/3}$, $\beta_3 = \zeta_3^3 5^{1/3}$, where $\zeta_3 = e^{2\pi i /3}$.\\\\
Note that we can take $c = 1$, and it has the property that
\[\alpha_i + c\beta_k \neq \alpha_1 + c\beta_1, 1 \leq i \leq 2, 2 \leq k \leq 3\]
SInce LHS is not real and RHS is always real.\\\\
SO we can take
\[\theta = \sqrt{2} + 5^{1/3}\]
So we have that
\[\Qbb(\sqrt{2}, 5^{1/3}) = \Qbb(\sqrt{2} + 5^{1/3})\]
\end{example}

\subsection{Conjugates of algebraic numbers}

The embeddings of a number field in $\Cbb$ play a fundamental role in Number Theory.

\begin{theorem}[Pg. 40]
Let $K = \Qbb(\theta)$ be a number field of degree $n$ over $\Qbb$. Then there exists exactly $n$ distinct field embeddings of $K$ into $\Cbb$. (Label these $\sigma_1, ..., \sigma_n: K \to \Cbb$).\\\\
Moreover, $\sigma_i(\theta)$ are the zeroes in $\Cbb$ of the minimal polynomial of $\Qbb(\theta)$ over $\Qbb$.
\end{theorem}

\begin{proof}
Suppose $\sigma: K \to \Cbb$ is an embedding, then $\sigma$ has to fix $\Qbb$ (becuse it sends $1$ to $1$), so field automorphisms permutes the roots of the minimal polynomial, so $\sigma(\theta)$ is a root of the minimal polynomial over $\theta$ over $\Qbb$.\\\\
(We note that if two field isomorphism agrees on $\theta$, they are the same isomorphism, so it is injective).\\\\
Now conversely, suppose I have two field automorphisms that sends $\theta$ to the same root, then we wish to show that they are the same field automorphism.\\\\
Indeed, from 1540 we have a unique field isomorphism (so injective)
\[\Qbb(\theta) \cong \Qbb(\theta_i), \theta \mapsto \theta_i\]
THis isomorphism comes from (1540 - Proposition 8.6 from Silverman)
\[\Qbb(\theta) \cong \frac{\Qbb[x]}{(f)}, \Qbb(\theta_i) \cong \frac{\Qbb[x]}{(f)}\]
So there's a biection between the roots of $f$ and the embeddings of $K$ into $\Cbb$.
\end{proof}

\subsection{Discriminant of bases, Vandermount determinant}

\begin{definition}
Let $K = \Qbb(\theta)$ be a number field of degree $n$, and let $\{\alpha_1, ..., \alpha_n\}$ be a basis of $K$ as a vector space over $\Qbb$. Let $\sigma_i: K \to \Cbb$, $1 \leq i \leq n$ be embeddings of $K$ into $\Cbb$.\\\\
Then, the \textbf{discriminant} of $\{\alpha_1, ..., \alpha_n\}$ is denoted as
\[\Delta[\alpha_1, ..., \alpha_n] = det(\sigma_i(\alpha_j))^2 = det\begin{bmatrix}
\sigma_1(\alpha_1) & ... & \sigma_1(\alpha_n)\\
\vdots & \ddots & \vdots\\
\sigma_n(\alpha_1) & ... & \sigma_n(\alpha_n)\\
\end{bmatrix}\]
If $\{\beta_1, ..., \beta_n\}$ is another basis, then for all $1 \leq k \leq n$,
\[\beta_k = \sum_{i = 1}^n C_{ik} \alpha_i, C_{ik} \in \Qbb\]
, where $det(C_{ik}) \neq 0$ since it's a map from basis to basis.
\end{definition}

\begin{proposition} The discriminant perserves linear transformation in the sense that, ie
\[\Delta[\beta_1, ..., \beta_n] = (det(C_{ik})^2 \cdot \Delta[\alpha_1, ..., \alpha_n]\]
\end{proposition}

\begin{proof}
Exercise!
\end{proof}

\begin{definition}
A \textit{Vandermount Matrix} is a matrix of the following form:
\[V = \begin{bmatrix}
1 & t_1 & ... & t_1^{n-1}\\
1 & t_2 & ... & t_2^{n-1}\\
\vdots & \vdots & \ddots & \vdots\\
1 & t_n & ... & t_n^{n-1}\\
\end{bmatrix}\]
\end{definition}

\begin{proposition}
Then the determinant of $V$ is
\[det(V) = \prod_{1 \leq i < j \leq n} (t_i - t_j)\]
\end{proposition}

\begin{theorem}[Pg. 42]
The discriminant of any basis for $K = \Qbb(\theta)$ is rational and non-zero.
\end{theorem}

\begin{proof}
The most natural choice of basis is $1, \theta, ..., \theta^{n-1}$, it suffices for us to prove that this basis is rational and non-zero, since $(det(C_{ik})^2$ is a non-zero rational numbers.\\\\
Write $\theta = \theta_1$, $\theta_1, ..., \theta_n$ for the conjugates of $\theta_1$ (ie. other roots of the minimal polynomial associated to $\theta$).\\\\
Then we have that
\[\Delta[1, ..., \theta^{n-1}] = (det(\theta_{i}^j))^2\]
Using a general observation, we note that we are calculating the determinant of a Vandermount Matrix, where in particular
\[(det(\theta_{i}^j))^2 = \prod_{1 \leq i < j \leq n} (\theta_i - \theta_j)^2 = disc(minpoly_\Qbb(\theta))\]
We note that $disc(minpoly_\Qbb(\theta))$ is surjective on $\Qbb^\times$, so we are done.
\end{proof}