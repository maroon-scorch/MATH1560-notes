\section{Last Lecture: Minkowski, Lagrange, and Waring Walk into a Bar}

\subsection{1. Four Squares Theorem and Waring's Problem}

\begin{theorem}[Lagrange, 1770]
Every non-negative integer can be written as a sum of four square integers.
\end{theorem}

\begin{remark}
In the same year Waring asseted, in his book, that for every integer $k \geq 2$, there is a $g(k)$ such that every non-negative integers can be written as the sum of at most $g(k)$ $k$-th powers.\\\\
So the Four Square Theorem is the same as saying $g(2) = 4$.\\\\
Waring then claimed that $g(3) = 9$ and $g(4) = 19$. Note that $23$ and $239$ each requires $9$ cubes, and $79$ required $19$ fourth powers.\\\\
The assertion was proven by Hilbert in 1909.\\\\
$g(3) = 9$ was proven by Wiefreich-Kamper in 1909, and $g(4) = 19$ was proved in $1986$ by Balasubramanian, Dress, and Deshuvilles.
\end{remark}

\begin{remark}
Another interesting question to ask is what is the ``Asymptotic version" of $g(k)$ is, which we denote as $G(k)$.\\\\
We know that
\[4 \leq G(3) \leq 7\]
\[G(4) = 16\]
So in general, the mileage differs on these.
\end{remark}

\subsection{2. Lattices and Minkowski's Theorem}

\begin{definition}
Let $e_1, ..., e_n$ be a set of basis vectors for $\Rbb^n$. Then the additive subgroup of $(\Rbb^n, +)$ generated by $e_1, ..., e_n$ is called a \textbf{lattice}.
\end{definition}

\begin{example}
$\Zbb^n$ is the most obvious lattice in $\Rbb^n$, generated by the standard basis.\\\\
$\alpha \Zbb^n$ for $0 \neq \alpha \in \Rbb$ is also an obvious choice.\\\\
Also things, like $\frac{1}{2} \Zbb \times \Zbb \subset \Rbb^2$ is also a lattice.
\end{example}

\begin{definition}
If $L$ is a lattice generated by $e_1, ..., e_n$ in $\Rbb^n$, then the \textbf{fundamental domain} of $L$ is the set
\[\{\sum_{i = 1}^n a_i e_i | a_i \in \Rbb, 0 \leq a_i < 1\}\]
\end{definition}

\begin{definition}
A set $X \subset \Rbb^n$ is convex if for all $x, y \in X$, 
\[\lambda(x) + (1 - \lambda)y \in X, \lambda \in [0, 1]\]
X is \textbf{symmetric} if $x \in X \implies -x \in X$.
\end{definition}

\begin{theorem}[Minkowski, pg. 140 of S + T]
Let $L$ be an ($n$-dimensional) lattice in $\Rbb^n$ with foundamental domain $T$, and let $X$ be a bounded, symmetric, convex subset of $\Rbb^n$. If
\[vol(X) > 2^n vol(T)\]
Then $X$ contains a non-zero point of $L$.\\\\
(We note that a symmetric and convex set necessarily centers around the origin).
\end{theorem}

\begin{remark}
If $T$ is the fundamental domain of standard basis in $\Rbb^2$, ie. $L = \Zbb^2$, then clearly $vol(T) = 1$ our statement says
\[vol(X) > 4\]
The intuition is that the four squares around the xy-axis has area 4, so $X$ has to contain a lattice point.
\end{remark}

\begin{lemma}
Let $f: \Rbb^n \to \Rbb^n$ be a linear transformation, let $X$ be symmetric and convex, then $f(X)$ is also symmetric and convex.
\end{lemma}

\begin{theorem}
Every non-negative integer can be written as a sum of four square integers.
\end{theorem}

\begin{proof}
We will first prove the statement for prime numbers, then we will extend this to all positive integers.\\\\
Clearly $2 = (1)^2 + (1)^2 + 0^2 + 0^2$ is a sum of four squares. Now we want to prove this for an odd prime.\\\\
We {\bf claim that} $r^2 + s^2 + 1 \equiv 0\mod\ p$ has a solution where $(r, s) \in \Zbb^2$, indeed this is because every element of $\Zbb/p\Zbb$ can be written as the sum of two squares, the main idea is to do this $p \mod 4$ and use Quadratic Residues, the details are left as exercise.\\\\
Select such an $r, s$. Now consider the lattice $\Lambda \subset \Zbb^4$ given by
\[\Lambda = A \Zbb^4\]
, where $A = \begin{pmatrix} p & 0 & r & s\\
0 & p & s & -r\\
0 & 0 & 1 & 0\\
0 & 0 & 0 & 1
\end{pmatrix}$.\\\\
If $\Vec{t} = (t_1, t_2, t_3, t_4) \in \Zbb^4$ and $\Vec{x} = (x_1, x_2, x_3, x_4)$ where $\Vec{x} = A \Vec{t}$. Then I claim that
\[x_1^2 + x_2^2 + x_3^2 + x_4^2 \equiv 0 \mod p\]
Indeed, we have that
\begin{align*}
    x_1^2 + x_2^2 + x_3^2 + x_4^2 &= (pt_1 + rt_3 + st_4)^2 + (pt_2 + st_3 - rt_4)^2 + (t_3)^2 + (t_4)^2 \mod p\\
    &\equiv (rt_3 + st_4)^2 + (st_3 - rt_4)^2 + t_3^2 + t_4^2 \mod p\\
    &\equiv (1 + r^2 + s^2)(t_3^2 + t_4^2) \mod p \tag*{Try to Compare Coefficients}\\
    &\equiv 0 \mod p \tag*{Since $1 + r^2 + s^3 \equiv 0 \mod p$}
\end{align*}
The idea now is that we will construct a ball around the origin that's small enough so that Minkowski's Theorem tell us that their sum has to be the prime $p$.\\\\
Recall that a $4$-dimensional ball of radius $R$ has volume
\[V(R) = \frac{\pi^2 R^4}{2}\]
Now we will choose $R$ such that
\[(i)\ 16p^2 < V(R) \implies 2p < \approx 1.11 R^2\]
\[(ii)\ R^2 < 2p\]
Thus we want $R$ such that
\[R^2 < 2p < \approx 1.11 R^2\]
So taking $R^2 = 1.9 p$ works. Let $X$ be the ball centered at the origin in $\Rbb^4$ of radius $R = \sqrt{1.9 p}$. Let $T$ be the fundamental domain of $\Lambda = A\Zbb^4$.\\\\
Then
\[vol(T) = det(A) = p^2\]
Since from (i), we have that
\[V(R) > 16p^2 = 2^4 vol(T)\]
So we can apply Minkowski's Theorem, $X$ contains some non-zero lattice point of $\Lambda$ say $(x_1, x_2, x_3, x_4) \neq (0, 0, 0, 0)$.\\\\
Furthermore, we know that $x_1^2 + x_2^2 + x_3^2 + x_4^2 = np$ for some $n \in \Nbb$. Howeverm subce $R^2 < 2p$, the square of norm of the vector is less than $2p$, so we have to have the case that
\[x_1^2 + x_2^2 + x_3^2 + x_4^2 = p\]
Now for the case of natural numbers, we claim that for any two positive integers $a, b$ that can be written as a sum of 4 squares, $ab$ is also a sum of 4 squares. Indeed, this follows from property of Quarternions.\\\\
More explicitly,
\[(a^2 + b^2 + c^2 + d^2)(A^2 + B^2 + C^2 + D^2)\]
\[ = (aA - bB - cC - dD)^2 + (aB + bA + cD - dC)^2 + (aC - bD + cA + dB)^2 + (aD + bC - cB + dA)^2\]
\end{proof}