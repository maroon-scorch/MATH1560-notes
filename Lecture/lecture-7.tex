\section{Lecture 7 (February 17th) - Special Integers:}

\subsection{Fermat and Mersenne primes}
\begin{example}
Many small primes are of the form $2^m \pm 1$ for some natural number $m$, eg.
\[3, 5, 7, 17, 31\]
We deal with the $+1$ and $-1$ cases separately.
\end{example}

\begin{lemma}
If $2^m + 1$ is prime, then $m = 2^n$ for some $n \geq 0$
\end{lemma}

\begin{proof}
We will prove using the contrapositive. Suppose that $m$ is not a power of 2, then $2^m + 1$ cannot be prime.\\\\
Since $m$ is not a power of 2, we can write $m$ as $m = 2^n \cdot q$, for some odd $q > 1$.\\\\
Consider the polynomial
\[f(t) = t^q + 1\]
Clearly $t = -1$ is a root of $f(t)$ as $q$ is odd, so
\[f(t) = (t + 1)g(t)\]
, where $deg(f) = q > 1$.\\\\
Consider when $x^m + 1 = f(x^{2n})$, then
\[x^m + 1 = f(x^{2n}) = (x^{2n} + 1)(g(x^{2n}),\ m > 2^n\]
When $x = 2$, then we are done, where $2^{2^n} + 1\ |\ 2^m + 1$
\end{proof}

\begin{definition}
Numbers of the form $2^{2^n} + 1$ are called \textit{Fermat Numbers}. Fermat Numbers that are also prime are called \textit{Fermat Primes}.
\end{definition}

\begin{example}
The first few Fermat Numbers happen to be prime:
\[3, 5, 17, 257, 65537\]
Fermat conjectured that all Fermat Numbers are primes, in fact the 5 numbers shown here were the only numbers he was able to confirm is prime.\\\\
Euler managaed to compute $2^{2^5} + 1$ and showed that $641$ divides said number.\\\\
So far no Fermat Numbers after the 5th Fermat number is prime.
\end{example}

\begin{lemma}
If $m > 1$, $a \geq 2$, and $a^m - 1$ is prime, then $a = 2$, and $m$ has to be prime.
\end{lemma}

\begin{proof}
Suppose $m$ is composite, so $m = n \cdot k$, $1 < n, k < m$. Then we have that
\[a^m - 1 = (a^k)^n - 1 = (a^k - 1)(a^{k(n-1)} + .... + 1)\]
Since $a$ is at least 2, so we have a proper divisor.\\\\
Therefore, $m$ has to be prime.\\\\
Now if $a > 2$, then $a^m - 1 = (a - 1)(a^{m - 1} + ... + 1)$. This factorization is only trivial when $a = 2$.
\end{proof}

\begin{definition}
Integers of the form $2^p - 1$ where $p$ is a prime are called \textit{Mersenne numbers}. \textit{Mersenne numbers} that are prime are called \textit{Mersenne primes}.
\end{definition}

\begin{remark}
Mersenne was a contemporary of Fermat's. We currently have 51 Mersenne primes, and we are unclear that whether or not there's an infinite number of Mersenne primes.\\\\
The largest Mersenne prime known so far (also the largest known prime) is:
\[2^{82,589,933} - 1\]
\end{remark}

\begin{definition}
$n \in \Zbb_+$ is perfect if
\[n = \sum_{d | n, d < n} d\]
\end{definition}

\begin{proposition}
If $n = 2^{p-1}(2^{p} - 1)$ where $p \in \Zbb_+$, $p$ is prime and $2^{p} - 1$ is prime, then $n$ is perfect.
\end{proposition}

\begin{proof}
The function $\sigma(n) = \sum_{d | n} d$ is a multiplicative function.\\\\
So if $n = 2^{p-1}(2^p - 1)$, then $\sigma(n) = \sigma(2^{p-1})\sigma(2^p-1)$ as they are coprime.\\\\
Then $\sigma(2^{p-1}) = \frac{2^p-1}{2-1} = 2^{p} - 1$,\\\\
Then $\sigma(2^{p} - 1) = 2^{p}$ as $2^{p} - 1$ is prime.\\\\
So we have that $\sigma(n) = 2n$, so we are done.
\end{proof}

\begin{proposition}
If $n \in \Zbb_+$ is an even perfect number, then $n = 2^{p-1}(2^p-1)$ where $p$ and $2^p-1$ are both prime.
\end{proposition}

\begin{proof}
The proof is trivial and is left as an exercise for the reader.
\end{proof}

\begin{theorem}[Euclid-Euler Theorem]
The even perfect numbers are in bijective correspondence to the Mersenne primes. Specifically, for any Mersenne prime $2^p - 1$, the map
\[2^p-1 \mapsto 2^{p-1}(2^p-1)\]
is the desired bijection.
\end{theorem}

\subsection{Pseudoprimes and Carmichael Numbers}

\begin{theorem}[Wilson's Theorem]
If $p$ is a prime, then
\[(p - 1)! \equiv -1\ (mod\ p)\]
\end{theorem}

\begin{proposition}
The converse of Wilson's Theorem is also true: if $n$ is a positive integer such that $n \geq 2$ and
\[(n - 1)! \equiv -1\ (mod\ n)\ \ (*)\]
Then $n$ is prime.
\end{proposition}

\begin{remark}
You can think of $(*)$ as a rudimentary primality test. It is not a great primality test, but it does give an explicit formula for primes
\[f(n) = \lfloor \frac{n!\ mod(n+1)}{n}\rfloor (n-1) + 2\]
We can also improve the test here.\\\\
Recall Fermat's Little Theorem: If $p \in \Zbb_+$ is prime and $a \in \Zbb$ is an integers, then
\[a^p \equiv a\ (mod\ p)\]
Thus take $n \in \Zbb_+$, if 
\[a^n \not \equiv a\ (mod\ n)\]
for some $a \in \Zbb_+$, then $n$ is composite.\\\\
Example,
\[2^n \not \equiv 2\ (mod\ n) \implies n \text{ is composite}\]
\end{remark}

\begin{example}
The converse of said test is unfortunately not true.\\\\
\[2^10 = 1024 \equiv 1\ (mod\ 341)\]
So $2^{341} = 2^{10}^{34} \cdot 2 \equic 2\ mod\ 341$.\\\\
But $341 = 11 \cdot 31$ and is thus compositie.\\\\
We say that $341$ is a pseudo-prime base 2.
\end{example}

\begin{definition}
We call $n$ a {\bf pseudo-prime} to the base $a$ if $n$ is composite and happens to satisfy
\[a^n \equiv a\ (mod\ n)\]
\end{definition}

\begin{remark}
Sadly, it is not true that every composite $n$ is a pseudo-prime base $a$ for all $a \in \Zbb$. The smallest composite number that is a pseudo-prime base $a$ for all $a \in \Zbb$ is $561$.
\end{remark}

\begin{definition}
A positive integer $n$ is called a {\bf Carmichael number} if $n$ is composite, and
\[a^n \equiv a\ (mod\ n), \forall a \in \Zbb\]
\end{definition}

\begin{proposition}
If a composite $n$ is not a Carmichael number, then at least half of the congruence classes $a \in (\Zbb/n\Zbb)^\times$ are such that $n$ is not a pseudo-prime to base $a$.
\end{proposition}

\begin{proof}
Suppose $n$ is a pseudo-prime to the base $a_1, a_2, ..., a_r \in (\Zbb/n\Zbb)^\times$, and suppose there exist $a$ such that ($a$ exists since $n$ is not a Carmichael Number):
\[a^n \equiv a\ (mod\ n)\]
Then for all $i$, 
\[(a \cdot a_i)^{n-1} = a^{n-1}a_i^{n-1} \equiv a^{n-1}\ mod(n) \nequiv 1\ mod(n)\]
Where in last inequality holds because if they do equal then $a^n \equiv\ a\ (mod(n))$prime, which gives a contradiction.\\\\
This means that $n$ is not a pseudo-prime to the bases
\[a \cdot a_1, a \cdot a_2, ..., a \cdot a_r\]
\end{proof}