\section{Lecture March 24th}

\subsection{Algebraic Integers ctd.}

 \begin{definition}
 A complex number $x$ that satisfies $f(x) = 0$ for a non-constant monic polynomial $f(t) \in \Zbb[x]$ is called an \textit{algebraic integer}.
 \end{definition}
 
 \begin{definition}[Equivalent Definition of Algebraic Integers]
 An algebraic integer is an algebraic number whose minimal polynomial over $\Qbb$ has integer coefficients.\\\\
 The set of algebraic integers is denoted by $\overline{\Zbb}$ and $\overline{\Zbb} \subset \overline{\Qbb}$. In fact, $\overline{\Zbb}$ form a subring of $\overline{\Qbb}$.
 \end{definition}
 
 \begin{lemma}[pg. 44 S+T]
 Let $\theta \in \Cbb$, then $\theta$ is an algebraic integer if and only if the additive group generated by all powers of $1, \theta, \theta^2, ... $ is in fact finitely generated. (ie. $\Zbb[\theta]$ is finitely generated $\Zbb$-module).
 \end{lemma}
 
 \begin{proof}
 In the forward direction, suppose $\theta \in \overline{\Zbb}$, then clearly there exist some non-constant polynomial $f(x)$ integer coefficients such that
 \[f(\theta) = \theta^n + a_{n-1}\theta^{n-1} + ... + a_0 = 0\]
 Then we claim that every power of $\theta$ lies in the additive group generated by $1, \theta, ..., \theta^{n-1}$, whic we will denote as $\Gamma$ (pretty obvious using Euclidean Division).\\\\
 Suppose inductively that $m \geq n$, and that $1, \theta, ..., \theta^m \in \Gamma$, then we can write
 \[\theta^{m+1} = \theta^{m + 1 - n} \theta^n = \theta^{m + 1 - n}(-a_{n-1}\theta^{n-1} - ... - a_0) = -a_{n-1}\theta^m - (\text{lower degree terms})\]
 , then apply the inductive htpothesis, we are done.\\\\
 Conversely, suppose every power of $\theta$ lies in a finitely generated additive group $G$. Then the subgroup $\Gamma$ of $G$ generated by $1, \theta, \theta^2, ....$ (countably many) must also be finitely generated since $\Zbb$ is a Noetherian ring so any submodule of a finitely generated $\Zbb$-module is finitely generated.\\\\
 Let $v_1, ..., v_n$ be generators of $\Gamma$ (WLOG assume are non-zero), then each $v_i \in \Zbb[\theta]$, and we also have that for all $i$
 \[\theta \cdot v_i \in \Zbb[\theta]\]
 Hence, there exist integers $b_{ij}$ such that
 \[\theta v_i = \sum_{j = 1}^n b_{ij} v_j\]
 This gives usa system of linear equations:
 \[(b_{11} - \theta)v_1 + b_{12}v_2 + ... + b_{1n}v_n = 0\]
 \[b_{21}v_1 + (b_{22} - \theta)v_2 + ... + b_{2n}v_n = 0\]
 \[\vdots\]
 \[b_{n1}v_1 + b_{n2}v_2 + ... + (b_{nn} - \theta)v_n = 0\]
 Then we can see that $v_1, ..., v_n$ is a solution to this system of linear equation. Let $A$ be the coefficient matrix, since it has a non-trivial solution, so $Ax = 0$ for some non-zero vector $x$, so $A$ is not invertible.\\\\
 Thus $det(A) = 0$, we can write $A = B - \theta I$, and $det(B - xI)$ is a monic characteristic polynomial with integer coefficients and root $x = \theta$, so we have that $\theta$ is an algebraic integer. 
 \end{proof}
 
\begin{lemma}[Weaker Statement of Above]
$\theta \in \Cbb$ is algebraic if and only if the additive subgroup generated by $1, \theta, \theta^2, ...$ is in fact generated by $1, \theta, \theta^2, ..., \theta^{n-1}$ for some $n$.
\end{lemma}

\begin{theorem}
$\overline{\Zbb}$ is a subring over $\overline{\Qbb}$
\end{theorem}

\begin{proof}
 Suppose that $\theta, \phi \in \overline{\Zbb}$, we wish to show that $\theta + \phi, \theta \phi \in \overline{\Zbb}$.\\\\
 By the Lemma, all powers of $\theta$ lie in a finitely generated $\Gamma_\theta$ of $\Cbb$ and similarly all powers of $\phi$ lie in a finitely generated subgroup $\Gamma_\phi$ of $\Cbb$.\\\\
 We observe that all powers of $\theta + \phi$ and $\theta \phi$ are integer linear combinations of the elements
 \[\theta^r \phi^s \in \Gamma_\theta \Gamma_\phi\]
 , where $\Gamma_\theta \Gamma_\phi$ is definied to be the additive group generated by $v_iw_j$, $1 \leq i \leq n, 1 \leq j \leq m$, where $v_i$ and $w_j$ are generators for $\Gamma_\theta$ and $\Gamma_\phi$ respectively.\\\\
 Thus, $\Gamma_\theta \Gamma_\phi$ are finitely generated, then by our Lemma above, we have that $\theta + \phi$ and $\theta \phi$ are both algebraic integers.
\end{proof}

\begin{theorem}[p. 44 or 45]
Let $\theta \in \Cbb$ satisfy a monic polynomial equation with coefficients in $\overline{\Zbb}$, then $\theta$ is indeed an algebraic integer.
\end{theorem}

\begin{proof}
One imitates the proof of the forward direction in our previous lemma, applying a bit of module theory.
\end{proof}

\subsection{Ring of integers of a number field}

\begin{definition}
If $K$ is a number field, then the set
\[\mathcal{O}_k = K \cap \overline{\Zbb}\]
is called the ring of integers of $K$. We note that $\mathcal{O}_k$ is the intersection of two subrings and is thus itself also a ring.
\end{definition}

\begin{remark}
This is analogous to the relationship between integers and rationals. In fact, the field of fraction of $\mathcal{O}_k$ is $K$.
\end{remark}

\begin{lemma}
Let $\alpha \in K$, then $c\alpha \in \mathcal{O}_k$ for some $c \in \Zbb$.
\end{lemma}

\begin{proof}
Suppose $\alpha \in K$, let $f(x) = minpoly_{\Qbb}(\alpha)$ and $deg(f) = n$.\\\\
Let $0 \neq c \in \Zbb$, let
\[g_c = c^n f(x/c)\]
Observe that for each $\alpha_i$ root of $f$, $c\alpha_i$ is a root of $g_c$.\\\\
We also note that $g_c$ is monic, and we can choose $c$ to be the least common multiple of all the denominators, then it in fact has integer coefficients. So $g_c$ is a monic integer polynomial with root $c \alpha$.
\end{proof}

\begin{corollary}
If $K$ is a number field, then $K = \Qbb(\theta)$ for some algebraic integer $\theta$. (as opposed to just some algebraic number)
\end{corollary}

\begin{proof}
Apply Lemma above and we can always choose our primitive element to be multipled by that $c$.
\end{proof}

\begin{remark}[WARNINGS! (46 - 47)]
Although it is often that case that if $K = \Qbb(\theta)$ with $\theta \in \overline{\Zbb}$, then $\mathcal{O}_K = \Zbb[\theta]$. This need not be true!\\\\
For example, suppose $K = \Qbb(\sqrt{5})$, then
\[\Zbb[\sqrt{5}] \subsetneq \mathcal{O}_K\]
In fact, $\Zbb[\frac{1 + \sqrt{5}}{2}] = \mathcal{O}_k$.\\\\
It in fact gets worse than this, $\mathcal{O}_k$ need not be of the form $\Zbb[\theta]$ fr some $\theta \in \overline{\Zbb}$. Dedekind first found a counter-example in 1871, with the example is
\[K = \Qbb(\theta), \theta \text{ a root of } x^3 - x^2 - 2x - 8\]
Number fields where $\mathcal{O}_k$ can be represented as $\Zbb$ adjoin some algebraic integer are called monogenic number fields.
\end{remark}