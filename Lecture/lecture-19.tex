\section{Lecture April 21st}

\subsection{Last Time:}

\begin{itemize}
    \item Discussed Fractional Ideals
    \item Introduced inverses to non-zero fractional ideals in $\mathcal{O}_K$ and more generally Dedekind Domains
    \item Proved $\mathcal{O}_K$ was a Dedekind Domain
    \item Provided an analog for prime factorization in $\mathcal{O}_K$, that every non-zero ideal of $\mathcal{O}_K$ factors uniquely as a product of primne ideals. (And analogously to all Dedekind Domains)
\end{itemize}

\subsection{This Lecture:}

A major topic/theme in classical algebraic number theory is the factorzation of ideals in $\mathcal{O}_K$ that are generated by primes in $\Zbb$.

\begin{definition}
Let $p \in \Zbb_+$ be a positive prime number, and let $K$ be a number field,
\begin{itemize}
    \item 1) We say $p$ \textbf{ramifies} in $K$ (or $\mathcal{O}_K$) if for 
    \[(p) := p \mathcal{O}_K = p_1^{e_1}p_2^{e_2}...p_r^{e_r}\]
    , where $p_1, ..., p_r$ are pairwise distinct prime ideals, and there's some $e_i \geq 2$
    \item 2) We say that $p$ is \textbf{totally ramified} if 
    \[(p) := p_1^n\]
    , where $n = [K: \Qbb]$, and $p_1$ is a prime ideal
    \item 3) We say that $p$ is \textbf{inert} if $(p)$ is a prime ideal of $\mathcal{O}_K$, ie.
    \[(p) = p_1\]
    \item 4) We say that $p$ is \textbf{totally split} if $(p) = p_1p_2...p_n$, $p_i$ pairwise distinct, $n = [K: \Qbb]$
\end{itemize}
Note that this is not an exhaustive list of prime factorizations in $\mathcal{O}_K$, but note that in a Quadratic Extension, this is indeed an exhaustive list.
\end{definition}

\begin{example}
Here are some examples of prime factorization:
\begin{itemize}
    \item 2 is totally ramified (and thus ramified) in $\Zbb[i]$ (ring of integers of $\Qbb(i)$), n particular
\[(2) = (1 + i)(1 - i) = (1 + i)^2\]
    \item 3 is inert in $\Zbb[i]$ since $(3)$ is a prime ideal in $\Zbb[i]$
    \item In $\Zbb[i]$, we can write $5 = (1 + 2i)(1 - 2i)$, which are both prime in $\Zbb[i]$. However, $1 + 2i, 1 - 2i$ aren't associates, because the only units of $\Zbb[i]$ are $\pm 1, \pm i$, and none of them make the cut.\\\\
    Thus, $5$ is totally split in $\Zbb[i]$.
\end{itemize}
\end{example}

\noindent There are two structural questions we want to answer:
\begin{itemize}
    \item 1) Which primes of $\mathcal{O}_K$ ramify?
    \item 2) How do individual rational primes (primes in $\Zbb$) factor in $\mathcal{O}_K$
\end{itemize}

\begin{theorem}
$p$ is ramified in $K$ if and only if $p | Disc(K)$
\end{theorem}

\begin{proof}
We will prove this in the monogenic case, ie. when $\mathcal{O}_K$ is of the form $\Zbb[\theta]$ for some $\theta \in \mathcal{O}_K$, but note that this holds for any general number field.\\\\
The main ideal of the proof is to study the facotrization of $f$ modulo $p$ where $f = minpoly_\Qbb(\theta)$.\\\\
Suppose $K$ is monogenic, with $\mathcal{O}_K = \Zbb[\theta]$, and let $p \in \Zbb_+$ be prime and let $f = minpoly_\Qbb(\theta)$.\\\\
Since $Disc(K) = \Delta[1, \theta, ..., \theta^{n-1}] = disc(f)$, we only need to show that 
\[p | disc(f) \iff p \text{ ramifies in } K\]
Let $p \mathcal{O}_K = p_1^{e_1}...p_r^{e_r}$ be the prime factorzation of $p \mathcal{O}_K$, then we note that
\[\mathcal{O}_K/(p) \cong \mathcal{O}_K/p_1^{e_1} \times ... \times \mathcal{O_K}/p_r^{e_r}\]
, by the Chinese Remainder Theorem. We note this is because $p_i + p_j \mathcal{O}_K$ for all $i \neq j$.\\\\
On the other hand, we note that
\[\mathcal{O_k}/(p) = \frac{\Zbb[\theta]}{p\Zbb[\theta]} \cong \frac{\Zbb[x]}{(p, f(x))}\]
This isomorphism comes from the fact that firstly:
\[\Zbb[\theta] \cong \frac{\Zbb[x]}{(f(x))}\]
Second, in general if $\frac{\frac{R}{(a)}}{(b)} \cong \frac{R}{(a, b)}$, 
Thus, we have that
\[\mathcal{O_k}/(p) = \frac{\Zbb[\theta]}{p\Zbb[\theta]} \cong \frac{\Zbb[x]}{(p, f(x))} \cong \frac{(\Zbb/p\Zbb)[x]}{(\overline{f}(x))}\]
If $\overline{f}(x) = \overline{\pi}_1(x)^{f_1} \overline{\pi}_2(x)^{f_2} ... \overline{\pi}_s(x)^{f_s}$, then 
\[(\Zbb/p\Zbb)[x]/(\overline{f}) \cong \frac{\mathbb{F}_p[x]}{\overline{\pi}_1^{f_1}} \times ... \times \frac{\mathbb{F}_p[x]}{\overline{\pi}_s^{f_s}}\]
(We note we can use the CRT again because $\Fbb_p[x]$ is an Euclidean Domain so gcd of two irreducible polynomial here is the constant 1 polynomial).\\\\
We note that for $\pfrak_1$ for $\mathcal{O}_K/(p)$, we have a chain of ideals
\[p_1^{e_1} \subsetneq p_1^{e_1 - 1} \subsetneq ... \subsetneq p_1 \subsetneq \mathcal{O}_K\]
So in other words, $\frac{p_1}{p_1^{e_1}}$ is a (in fact unique) maximal ideal of $\mathcal{O}_K/(p_1^{e_1})$.\\\\
Now we claim that if $R \cong R_1 \times R_2$ and $I \subset R_1$ is maximal in $R_1$, then $I \times R_2$ is maximal in $R$. So we know that $\mathcal{O}_K/(p)$ has a total of $r$ maximal ideals from one side (due to uniqueness), and the other side shows us it has a total of $s$ maximal ideals, so we have that $r = s$.\\\\
Start with the ideal $J_1 = I_1 \times \frac{\mathcal{O}_K}{p_2^{e_2}} \times ... \times \frac{\mathcal{O}_K}{p_r^{e_r}}$ where $I_1$ is maximal in $\mathcal{O}_K/p_1^{e_1}$, and then we descend from $I_1$ down all the way to the zero ideal for each $J_k$ (we note this terminates because $\frac{\mathcal{O}_K}{(p)}$ is a finite ring) - this is kind of the crux of the ``nilpotent argument" that shows that after appropriate reordering, we have that $e_i = f_i$ for all $i$ as well.\\\\
Hence we see that some $e_i \geq 2$ if and only if $\overline{f} mod p$ has a repeated root $f_i \geq 2$, so $\overline{f}$ is not separable. But this is equivalent to saying that $disc(f) \cong 0\ mod\ p$, so $p | disc(f)$
\end{proof}