\section{Lecture April 12th}

\subsection{Ideals & Fractional Ideals:}

\begin{definition}
Let $R$ be a commutative ring with unity. Recall that if $I, J$ are ideals of $R$, then 
\[I + J := \{a_i + b_j | a_i \in I, b_j \in J\}\]
is also an ideal, and
\[IJ := \{\sum_{i = 1}^n a_ib_i | a_i \in I, b_i \in J\}\]
\end{definition}

\begin{remark}
The naive definition of product of ideals does not work ie. if we define
\[IJ = \{a_ib_j | a_i \in I, b_j \in J\}\]
is not necessarily ideal. For example, in $\Zbb[x]$, take $I = (2, x), J = (3, x)$
\end{remark}

\begin{definition}
Let $K$ be a number field. An ideal of $\mathcal{O}_K$ is sometimes called an \textbf{integral ideal}. This is to contrast them with fractional ideals.
\end{definition}

\begin{definition}
A \textbf{fractional ideal} of $\mathcal{O}_K$ is a set of the form
\[C^{-1}\bfrak\]
, where $\bfrak$ is an ideal of $\mathcal{O}_K$, and $c$ is a non-zero element of $\mathcal{O}_k$
\end{definition}

\begin{example}
The fractional ideals of $\Zbb$ are of the form $r\Zbb$ where $r \in \Qbb$. For instance,
\[\frac{2}{5} \Zbb\]
is a fractional ideal of $\Zbb$.
\end{example}

\begin{remark}
We note that if $\mathcal{O}_k$ is a PID, then the fractional ideals are really easy to describe, specifically:
\[C^{-1}(d) = C^{-1}d \mathcal{O}_k\]
Let $\alpha = C^{-1} d$, then this is just the ideal generated by $\alpha$ in $\mathcal{O}_k$. This need not hold outside of a UFD.
\end{remark}

\begin{definition}[Addition and Multiplication of Fractional Ideals]
If $\afrak, \bfrak$ are fractional ideals, then 
\[\afrak \bfrak = \{\text{finite sum } \sum a_i b_j | a_i \in \afrak, b_j \in \bfrak\}\]
The definition for $\afrak + \bfrak$ is obvious. If
\[\afrak_1 = c_1^{-1} \bfrak_1, \afrak_2 = c_2^{-1} \bfrak_2\]
, where $\bfrak_1, \bfrak_2$ are integral ideals, then
\[\afrak_1 \afrak_2 = (c_1c_2)^{-1} \bfrak_1 \bfrak_2\]
, which is a fractional ideal.\\\\
The multiplication (of fractional ideals) is associative and commutative, with the unit ideal $\mathcal{O}_k$ as the multiplicative identity.\\\\
Thus, the set of non-zero fractional ideals form an abelian monoid under multiplication.
\end{definition}

\noindent \textbf{What about inverses?}

\begin{theorem}[pg. 109 of S+T]
The non-zero fractional ideals of $\mathcal{O}_k$ form a group under multiplication. (Note this usually doesn't hold unless we are in a Dedekind Domain)
\end{theorem}

\begin{proof}
For each integral ideal $\afrak \subset \mathcal{O}_k$, define
\[\afrak^{-1} = \{x \in K | x \afrak \subset \mathcal{O}_k\}\]
Clearly $\mathcal{O}_k \subset \afrak^{-1}$. If $\afrak$ is non-zero, then for any $0 \neq c \in \afrak$, we have that by definition
\[c\afrak^{-1} \subset \mathcal{O}_k\]
Fixing such a $c$, we have that $c \afrak^{-1} = \bfrak$ is in fact an ideal of $\mathcal{O}_k$.\\\\
This is because $c \afrak^{-1}$ is an $\mathcal{O}_k$-submodule of $\mathcal{O}_k$, ie. an ideal of $\mathcal{O}_k$).\\\\
Thus, $\afrak^{-1} = c^{-1} \bfrak$, so this is actually a fractional ideal.\\\\
By definition then, we have that
\[\afrak \afrak^{-1} = \afrak^{-1} \afrak \subset \mathcal{O}_k\]
Now we want to show that $\mathcal{O}_K \subset \afrak \afrak^{-1}$ (Blackboxed Momentarily pg. 110 - 112 of S + T, using the fact that $\mathcal{O}_K$ is a Dedekinf Domain)
\end{proof}

\begin{example}
The product of an ideal with its inverse need not be the whole ring. For example take $R = \Cbb[x, y]$ and consider its field of fraction $K$, then $R$ is our integral domain here and $K$ its field of fractions.\\\\
Now let $\afrak = (x, y)$, what is $\afrak^{-1}$, well since any element in $\afrak^{-1}$ has to multiply to every element of $\afrak$ into $\afrak$, we have that $\afrak^{-1} = R$, so
\[\afrak \afrak^{-1} \neq K\]
\end{example}

\begin{remark}
We can also extend the construction of inverses to fractional ideals using the exact same setup.
\end{remark}

\noindent Assuming this, we have shown that

\begin{theorem}[pg. 109 of S + T]
The non-zero fractional ideals of $\mathcal{O}_K$ form a group under multiplication.
\end{theorem}

\begin{proof}
We can deduce this quickly from the fact that integral ideals are invertible in $\mathcal{O}_K$.\\\\
Let $\afrak$ be a non-zero fractional ideal of $\mathcal{O}_K$. Then we have that
\[\afrak = c^{-1} \bfrak\]
$\bfrak$ is an integral ideal and $0 \neq c \in K$. Then define $a^\prime = c \bfrak^{-1}$, where $\bfrak^{-1}$ is the inverse of the integral ideal $\bfrak$. We clearly then have that
\[\afrak \afrak^\prime = \mathcal{O}_K\]
\end{proof}

\subsection{Dedekind Domain}

Recall: A prime ideal a commutative ring $R$ can be defined in a couple of different ways:

\begin{proposition}[Equivalent Definition of Prime Ideals]
The following definitions of a prime ideal $\pfrak$ in a commutative ring $R$ is:
\begin{itemize}
    \item 1) If for any two ideals $I, J$ of $R$ such that $IJ \subset \pfrak \implies I \subset \pfrak$ or $J \subset \pfrak$
    \item 2) For any two elements $a, b \in R$ such that $ab \in \pfrak \implies a \in \pfrak$ or $b \in \pfrak$
\end{itemize}
\end{proposition}

\noindent To prove unique factorization of non-zero ideals, we first need to prove $\mathcal{O}_K$ is a Dedekind Domain.

\begin{definition}
A chain of ideals is a sequence of inclusions:
\[I_1 \subset I_2 \subset ...\]
and for such a chain to terminate means that there exist some $N$ such that $I_n = I_N$ for all $n \geq N$.
\end{definition}

\begin{theorem}[pg. 109 - S + T]
The ring of integers $\mathcal{O}_K$:
\begin{itemize}
    \item a) is an integral domain
    \item b) is Noetherian (every ascending chain of ideals terminates, or equivalently every ideal is finitely generated)
    \item c) The ring of integers is integrally closed in its field of fractions, meaing if $\alpha \in Frac(\mathcal{O}_K) = K$ satisfies a monic polynomial equation with coefficients in $\mathcal{O}_K$, then $\alpha \in \mathcal{O}_K$.
    \item d) Every non-zero prime ideal of $\mathcal{O}_K$ is maximal.
\end{itemize}
We also note that a ring satisfying $(a) - (d)$ is known as a Dedekind Domain.
\end{theorem}