\section{Lecture March 8th}

\subsection{Proof of Quadratic Reciprocity}

\begin{example}
Determine if $219$ is a quadratic residue modulo $383$ (which is infact a prime number).\\\\
Since $383$ is prime, this is the same as asking
\[(\frac{219}{383}) = (\frac{3}{383}) \cdot (\frac{73}{383}) = \]
, we note that both $3$ is $3\ mod 4$ and $73$ is $1\ mod 4$, so
\[= -(\frac{383}{3}) (\frac{383}{73}) = -(\frac{2}{3}) \cdot (\frac{18}{73}) = (\frac{18}{73}) = (\frac{2}{23})\]
By the second supplemental law of quadratic residues, we have that
\[(\frac{2}{23}) = 1\]
\end{example}

\begin{remark}
Note that we must factor the top argument before beginning to flip using Quadratic Reciprocity.
\end{remark}

\noindent Now we will finally prove Quadratic Reciprocity. We present a proof using Gauss's Lemma.

\begin{theorem}
For $p, q \in \Zbb_+$, distinct odd primes,
\[(\frac{p}{q})(\frac{q}{p}) = (-1)^{\frac{(p-1)(q-1)}{4}}\]
\end{theorem}

\begin{proof}
Let $P = \{1, 2, ..., \frac{p-1}{2}\}$, $N = -P - \{-1, -2, ..., -\frac{p-1}{2}\}$, $Q = \{1, 2, ..., \frac{q-1}{2}\}$.\\\\
Write $\overline{P}, \overline{N}$ for $P\ mod\ p$ and $N\ mod\ p$ respectively, so that Gauss's Lemma gives us that
\[(\frac{q}{p}) = (-1)^{\mu}\]
, where $\mu = |q \overline{P} \cap \overline{N}|$.\\\\
In other words, $\mu$ is exactly the number of $x \in P$ such that
\[qx \equiv n\ mod\ p\]
, for some $n \in N$, and hence this $\mu$ is the number of $x \in P$ such that for some $y \in \Zbb$, (n has to lie strictly between $-p/2$ and $0$)
\[-\frac{p}{2} < qx - py < 0\]
We now specify more precisely which $y$ can possibly satisfy this condition for some $x \in P$.\\\\
Solving these two inequalities for $y$ gives
\[\frac{qx}{p} < y < \frac{qx}{p} + \frac{1}{2}\]
On the other hand, since $x \leq \frac{p-1}{2}$, for all $x \in P$, this means gives
\[y < \frac{qx}{p} + \frac{1}{2} \leq \frac{q(p-1)}{2p} + \frac{1}{2}\]
We note that $\frac{p-1}{p} < 1$, so
\[y < \frac{q(p-1)}{2p} + \frac{1}{2} < \frac{q}{2} + \frac{1}{2} = \frac{q+1}{2}\]
Thus we have that ($0 < y$ is implicit)
\[0 < y < \frac{q-1}{2}\ (*)\]
Since $y$ is an integer, we know that $y \in Q = \{1, 2, ..., \frac{q-1}{2}\}$.\\\\
Thus, we've shown that $\mu$ is the number of ordered pairs
\[(x, y) \in P \times Q,\ -\frac{p}{2} < qx - py < 0\]
{\bf Now we will switch the roles of p and q}, then we also have that 
\[(\frac{p}{q}) = (-1)^{n}\]
, where $n$ is the number of pairs
\[(y, x) \in Q \times P,\ -\frac{q}{2} < py - qx < 0\]
, where we note the condition above is exactly the number of pairs
\[(x, y) \in P \times Q, 0 < qx - py < \frac{q}{2}\]
Then we see that
\[(\frac{p}{q}) \cdot (\frac{q}{p}) = (-1)^{\mu + n}\]
, so our question is, what is $\mu + n$.\\\\
We note that the two conditions for $\mu$ and $n$ are mutually exclusive! So the number of $(x, y) \in P \times Q$ such that
\[-\frac{p}{2} < qx - py < 0, \text{ or } 0 < qx - py < \frac{q}{2}\]
is exactly $\mu + n$.\\\\
We note that it is impossible for $qx - py = 0$ since if $qx = py$ then that would imply $x$ is at least $q$ and $y$ is at least $p$, but $Q < q$ and $P < p$, so our condition above simplifies to
\[-\frac{p}{2} < qx - py < \frac{q}{2}\]
, and the number of $(x, y)$ satisfying this is still $P \times Q$.\\\\
Graphically, we are looking at the ordered pairs in a rectangular box $R = [1, p-1/2] \times [1, q-1/2]$ bounded by the two lines
\[qx - py = -p/2, qx - py = q/2\].\\\\
So we are counting the number of lattice points in $R$ bounded by the two lines, we will indicate $A, B$ as the complement of the shaded region in $R$.\\\\
If $\alpha$ is the number of integer points in $A$ and $\beta$ be the number of integers points in $B$, then we have that
\[\mu + n = (\frac{p-1}{2})(\frac{q-1}{2}) - (\alpha + \beta)\]
As long as we show that the equation above agrees modulo 2, then we are good, so we want to show that $\alpha + \beta$ is 0 modulo 2.\\\\
We will do this by showing that $\alpha = \beta$.\\\\
Let $\rho$ be the rotation given by
\[\rho(x, y) = (\frac{p+1}{2} - x, \frac{q+1}{2} - y)\]
, we note that $(\frac{p+1}{4}, \frac{q+1}{4})$ is the center of this rotation as they are fixed, and we claim that $\rho(R) = R$ as in that the rotation maps R to itself, specifically $\rho(A) = B$, $\rho(B) = A$.\\\\
To do the last part, let $\rho(x, y) = (x^\prime, y^\prime)$, one need to check 
\[qx - py < \frac{-p}{2} \iff qx^\prime - py^\prime > \frac{q}{2}\]
We omit the step, and this implies that
\[\rho(A) = B, \rho(B) = A\]
Clearly $\rho$ is a bijection and it preserves lattice points, so we have that $\alpha = \beta$.\\\\
Thus, we have that
\[\mu + n = \frac{(p-1)(q-1)}{4} - 2\alpha\]
So we have that
\[\mu + n \equiv \frac{(p-1)(q-1)}{4}\ (mod\ 2)\]
\end{proof}

\subsection{Jacobi Symbol}

The Jacobi symbol generalizes the Legendre Symbol.

\begin{definition}
Let $b$ be an odd positive integer, and let $a$ be any integer. Write
\[b = p_1p_2...p_m\]
, where $p_1, ..., p_n$ are primes (not necessarily distinct).\\\\
The symbol $(\frac{a}{b})$ defined by
\[(\frac{a}{b}) = (\frac{a}{p_1})...(\frac{a}{p_m})\]
is called the Jacobi Symbol. (The Legendre Symbol is a special case when $b$ is prime)
\end{definition}

\begin{proposition}[Basic Properties of Jacobi Symbol]
A few basic properties
\begin{itemize}
    \item $(\frac{a_1a_2}{b}) = (\frac{a_1}{b})(\frac{a_2}{b})$ (Follows from Multiplicativity of the Legendre Symbol)
    \item $(\frac{a}{b_1b_2} = (\frac{a}{b_1})(\frac{a}{b_2}))$ (Trivial from Definition)
\end{itemize}
So the Jacobi Symbol is totally multiplicative on the top and bottom.
\end{proposition}

\begin{example}
Big Warning:
\[(\frac{a}{b}) = 1 \not \implies \text{ a is quadratic residue modulo b}\]
For example $(2/15) = 1$, but 2 is not a quadratic residue modulo 15.\\\\
However,
\[(\frac{a}{b}) = -1 \implies \text{ a is a non-residue modulo b}\]
This follows from the fact that $(a/p_i) = -1$ for at least one of the prime factors of $b$.
\end{example}

\begin{proposition}[5.2.2 of the Text]
Let $b \in \Zbb_+$ be odd, then
\begin{itemize}
    \item a) $(\frac{-1}{b}) = (-1)^{(b-1)/2}$
    \item b) $(\frac{2}{b}) = (-1)^{(b^2-1)/8}$
    \item If $a, b \in \Zbb_+$ are odd, then
    \[(\frac{a}{b}) \cdot (\frac{b}{a}) = (-1)^{\frac{(a-1)(b-1)}{4}}\]
\end{itemize}
\end{proposition}