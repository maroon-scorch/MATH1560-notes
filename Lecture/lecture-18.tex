\section{Lecture April 14th}

Recall that last time we
\begin{itemize}
    \item Defined fractional ideals
    \item Defined the inverse of an integral ideal (and that the same definition holds for fractional ideal - with the exception of the zero ideal, which is fractional but not invertible)
    \item Briefly stated the definition of a Dedekind Domain
\end{itemize}

\begin{theorem}[pg. 109 - S + T]
The ring of integers $\mathcal{O}_K$:
\begin{itemize}
    \item a) is an integral domain
    \item b) is Noetherian (every ascending chain of ideals terminates, or equivalently every ideal is finitely generated)
    \item c) The ring of integers is integrally closed in its field of fractions, meaning if $\alpha \in Frac(\mathcal{O}_K) = K$ satisfies a monic polynomial equation with coefficients in $\mathcal{O}_K$, then $\alpha \in \mathcal{O}_K$.
    \item d) Every non-zero prime ideal of $\mathcal{O}_K$ is maximal.
\end{itemize}
We also note that a ring satisfying $(a) - (d)$ is known as a Dedekind Domain.
\end{theorem}

\begin{proof}
(a) is obvious, (c) was noted in a previous lecture.\\\\
For (b), we know that if $[K: \Qbb] = n$, then $\mathcal{O}_K$ is a free $\Zbb$-module of rank $n$ (ie. $\mathcal{O}_K$ is a free abelian group of a rank $n$).\\\\
If $\afrak$ is an ideal of $\mathcal{O}_K$, then $(\afrak, +)$ is free abelian of rank less than or equal to $n$ (See. Theorem 1.16 of S + T) So in other words, $(\afrak, +)$ is finitely generated as an $\mathcal{O}_K$ module (since $\Zbb \subset \mathcal{O}_K$, so every ideal of $\mathcal{O}_K$ is finitely generated with respect to $\mathcal{O}_K$.\\\\
For (d), let $\pfrak$ be a non-zero prime ideal of $\mathcal{O}_K$. Let $0 \neq \alpha \in \pfrak$. Then consider $N := N_{K/\Qbb}(\alpha) = \alpha_1 ... \alpha_n$, where $\alpha_1, ..., \alpha_n$ are conjugates of $\alpha$ and define $\alpha_1 = \alpha$.\\\\
We note that $N(\alpha) \in \Qbb$ since its fixed by all the filed embeddings. Moreover, $\alpha_2 ... \alpha_n = \frac{N(\alpha)}{\alpha} \in K$. In fact, $\alpha_2\alpha_3...\alpha_n \in \mathcal{O}_K$.\\\\
Thus, we have that $N(\alpha) \in \pfrak$ since $\pfrak$ is an ideal.\\\\
Thus $N(\alpha) \cdot \mathcal{O}_K \subset \pfrak$, which means that
\[\mathcal{O}_K/\pfrak \text{ is a quotient of } \mathcal{O}_K/N \mathcal{O}_K\]
Now, $\mathcal{O}_K/N \mathcal{O}_K$ is a finitely generated abelian group where every element has finite order (add it N times, goes to 0), so in particular $\mathcal{O}_K/N \mathcal{O}_K$ is a finite group.\\\\
Hence we have that $\mathcal{O}_K/\pfrak$ is finite group. Since $\pfrak$ is prime, $\mathcal{O}_K/\pfrak$ is an integral domain. But we know that any finite integral domain is in fact a field, so this means that $\pfrak$ is actually a maximal ideal.
\end{proof}

\begin{proposition}[pg. 112 S + T - Existence Claim]
Every non-zero ideal $\afrak \subset \mathcal{O}_K$ is a product of prime ideals.
\end{proposition}

\begin{proof}
Suppose not, then let $\afrak$ be a maximal element of the set of ideals of $\afrak$ that is NOT a product of prime ideals. Why does this exist?\\\\
Recall Zorn's Lemma: In a poset where every chain has an upperbound that's also in the poset, there's at least 1 maximal element in said poset.\\\\
The Noetherian property of $\mathcal{O}_K$ tells us every chain has an upperbound in that poset (some finite termination), so Zorn's Lemma gives us this maximal element.\\\\
Certainly $\afrak$ itself is not a prime ideal, but applying Zorn's Lemma to the poset of proper ideals containing $\afrak$ to conclude that $\afrak \subset \pfrak$ for some maximal ideal $\pfrak$, which is prime.\\\\
We have that $\mathcal{O}_K \subsetneq \pfrak^{-1} \subset \afrak^{-1}$ since inverses of integral ideals is inclusion reversing and $\afrak \subset \pfrak \subset \mathcal{O}_K$. (We note it respects proper containment since you can also take an inverse back)\\\\
It follows that 
\[\afrak \subsetneq \afrak \pfrak^{-1} \subset \afrak \afrak^{-1} = \mathcal{O}_K\ (*)\]
By the maximality of $\afrak$, we know that $\afrak \pfrak^{-1}$ is a product of prime ideals, but this means that
\[\afrak \pfrak^{-1} = \pfrak_2 ... \pfrak_r\]
, where $\pfrak_2, ..., \pfrak_r$ are prime, but then we have that
\[\afrak = \pfrak \pfrak_2 ... \pfrak_r\]
is a product of prime ideals, hence a contradiction.
\end{proof}

\begin{lemma}[pg. 113 S+T]
For ideals $\afrak, \bfrak$ of $\mathcal{O}_K$ (or more generally a Dedekind Domain), we have that 
\[\afrak | \bfrak \iff \afrak \supseteq \bfrak\]
(Note we say $\afrak | \bfrak$ if there exist an ideal $C$ such that $\bfrak = C \afrak$)
\end{lemma}

\begin{proof}
Suppose $\afrak | \bfrak$, clearly $\bfrak = C \afrak \subseteq \afrak$. Conversely, suppose that $\bfrak \subseteq \afrak$. Now let $\bfrak = \afrak(\afrak^{-1} \bfrak)$, where $\afrak^{-1} \bfrak$ is integral since $\afrak^{-1} \bfrak \subset \afrak^{-1} \afrak = \mathcal{O}_K$.
\end{proof}

\begin{theorem}[pg. 112]
Every nonzero ideal of $\mathcal{O}_K$ has a unique factorization as product of prime ideals.
\end{theorem}

\begin{proof}
The lemma above tells us that $\pfrak$ is prime if and only if $\pfrak | \afrak \bfrak \implies \pfrak | \afrak$ or $\pfrak | \bfrak$ (specifcally if helps the forward direction).\\\\
Suppose $\afrak = \pfrak_1 ... \pfrak_r = \qfrak_1 ... \qfrak_s$, as product of prime ideals (not necessarily distinct).\\\\
Then $\pfrak_1$ divides one of $\qfrak_j$ for some $j$, so since $\qfrak_j$ is maximal in $\mathcal{O}_K$, we have that $\pfrak_1 = \qfrak_j$.\\\\
Now we can go down the list from $1, ..., r$ because we can get rid of the previous $\pfrak_i$ using $\pfrak_i^{-1}$, and repeating the process tells us that this factorization is unique.
\end{proof}